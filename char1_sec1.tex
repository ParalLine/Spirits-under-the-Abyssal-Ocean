
 

\section{\color{blueriver}{\highlight{鲊の章}}}  
{\hfill 你知道吗?

\hfill 海月水母是蜇不死人的}


\def\cla{white}
\def\clb{bluesky}
\def\clc{blueriver}
\def\cld{yellow}

%\def\cla{darkred}
%\def\clb{pink}
%\def\clc{rose}
%\def\cld{yellowgr}

%\def\cla{darkpurple}
%\def\clb{redpink}
%\def\clc{purple}
%\def\cld{white}

%\def\cla{white}
%\def\clb{brownl}
%\def\clc{darkbrown}
%\def\cld{rice}

\vspace{10 ex}
\begin{center}
	
\dress{\cla}{\clb}{\clc}{\cld}\hspace{7.3 ex}\dress{\cla}{\clb}{\clc}{\cld}\hspace{7.3 ex}\dress{\cla}{\clb}{\clc}{\cld}\hspace{7.3 ex}\dress{\cla}{\clb}{\clc}{\cld}\hspace{7.3 ex}\dress{\cla}{\clb}{\clc}{\cld}

\vspace{4.4 ex}\dress{\cla}{\clb}{\clc}{\cld}\hspace{7.3 ex}\dress{\cla}{\clb}{\clc}{\cld}\hspace{7.3 ex}\dress{\cla}{\clb}{\clc}{\cld}\hspace{7.3 ex}\dress{\cla}{\clb}{\clc}{\cld}\hspace{7.3 ex}\dress{\cla}{\clb}{\clc}{\cld}

\vspace{4.4 ex}\dress{\cla}{\clb}{\clc}{\cld}\hspace{7.3 ex}\dress{\cla}{\clb}{\clc}{\cld}\hspace{7.3 ex}\dress{\cla}{\clb}{\clc}{\cld}\hspace{7.3 ex}\dress{\cla}{\clb}{\clc}{\cld}\hspace{7.3 ex}\dress{\cla}{\clb}{\clc}{\cld}
	
\end{center}

%\shirt{cyannew}{white}{pink2}{redpink}{lblue}{bluesky}{yellowgr}

\newday{\newmoon}
%\begin{tikzpicture}
%	% 设定砖块的大小
%	\def\brickwidth{1}
%	\def\brickheight{0.5}
%	
%	% 遍历整个砖块区域
%	\foreach \x in {0,1,...,6} { % x 轴范围
%		\foreach \y in {0,1,...,4} { % y 轴范围
%			% 计算 (x + y) 的和并判断是否为奇数
%			\ifodd\the\numexpr\x + \y\relax
%			% 如果是奇数,填充颜色1
%			\fill[red] (\x*\brickwidth, \y*\brickheight) rectangle +(\brickwidth, \brickheight);
%			\else
%			% 如果是偶数,填充颜色2
%			\fill[blue] (\x*\brickwidth, \y*\brickheight) rectangle +(\brickwidth, \brickheight);
%			\fi
%		}
%	}
%\end{tikzpicture}


\newpage
\newday{{\sunset}}
\subsection{水母的蜇刺}



“我说美月,你就跟我交往吧?我都追你那么久了,也该同意了吧~”

那是我高一刚开学不久后的一个傍晚,夕阳撒在教学楼上。我本该如往常一样回家,可那个男生在放学后将我拦在墙角逼问着。

我胆怯地低声回复:“那个…我…我要回家了…请…请放我走吧…”

男生:“诶?我也要回家啊~所以就不要浪费大家的时间了嘛~”

由于我本身并不爱说话,家里父母又时常不在家,所以大部分时间都更喜欢一个人安静地看书或者发呆。不知不觉就变得有些阴沉,开始不在意和周围同学的交流,或者应该说是变得不知道如何交流才更准确。或许这也是我初中三年都没交到什么朋友的理由吧?当然,直到现在我也没在高中班级上交到什么朋友,不过我并不介意。除了不知道某天起,开始注意到我的这个隔壁班的男生。他身上有着我很讨厌的味道。我不知道该如何让他打消念头,这反而变成了他变本加厉地纠缠着我的理由。

当我正在纠结不知道该怎么脱身时,另一个男生突然从身后不起眼的草丛里钻了出来,指着纠缠我的男生喊道:“喂!那边的家伙。吵到我睡觉了!”

我注意到那个男生的头发和肩上的衣服都沾着落叶,看来他真的在这里睡觉。

男生摆出一脸不良少年的神情朝着草丛里冒出来的男生嚷嚷着:“你是…谁啊?跟你有关系吗?!”

草丛里的男生:“这个嘛~我可以告诉你我姓内海(\japan{うつみ})。”

男生:“内海…?啊…?!不会是那个内海吧…?”

内海:“嗯…就是你想的那个内海。所以不想惹上麻烦的话,最好现在就从我的面前消失!”

男生:“切!真麻烦!今天碰到你真是倒霉,把我的好事都搅黄了!”

男生拎起甩在一旁的书包,灰溜溜地离开了。

而我完全不知道他们在说什么…

这名自称内海的男生拍了拍肩上的落叶,从草丛内跨出来走向我。当他来到我的面前,我自然地抬起头时,我才意识到他比自己高半个头。

内海:“你…最好拒绝他比较好。那种家伙不明确拒绝的话,会一直死缠烂打的。我见得多了。”

内海凝视着我的眼睛,我下意识地避开。

我:“那个…因为是隔壁的…同学…所以…不想拒绝…”

内海:“那你会接受吗?”


我拼命地摇了摇头:“因为…他身上有烟味…我不喜欢…”

“哦!原来如此,感谢你提供的线索!”内海一个人托着下巴嘀咕着,“嗯哼…或许这才是他想要躲着我的原因…?”



我用余光撇了撇眼前这个从草丛里冒出来、连头顶的落叶都没有掸落的男生,他也如刚才那个男生一般,敞开着衬衫最上方的扣子,没有系领带,喉结利落地暴露在外面。明明是如此松散烂漫的模样,却并不让人讨厌。

我:“那个…请问…你的名字…是…?”

内海:“哦!我叫内海侑(\japan{うつみ ゆう}),叫我内海就好了!不过真稀奇呐,你居然问我是谁?高一新生?”

我胆怯地点点头。

内海突然用手摸着下巴,皱了皱眉头,把脸靠近观察着我。

我被草丛男生不断靠近的身体给吓到,不知道他要干什么:“那个…内海…学长?”

内海突然发现了什么似的,径直地伸出手,把我额头前的长发撩起来。

内海:“果然呢,你前面的头发有点太长了!这可不符合学校的仪容要求啊!难怪会觉得你整个人都阴森森的。”

“诶…?!”我被内海突然的肌肤接触给吓得手臂乱挥。

内海的手捂着我的额头,我惊讶地看着眼前这个不可思议的男生。男生的手掌宽厚而温暖,配合着纤长的手指,即便扭头也无法轻易逃脱,从掌心传来的温度让人觉得像是要发烧了一般。他的眉毛浓密坚挺,眉尾却微微下垂,热情中带着一丝温柔,眼神清澈而明亮,仿佛如深海般,多看一眼便会坠入。理应是初见的男生,他的眼神却有一股莫名地亲切感。

内海注视了一会我的眼睛,似乎终于能得出某个结论:“你看,你的眼睛不是挺秀气的嘛!干嘛要遮起来呢?”

我红着脸连忙用双手抓紧内海的手,将它从我的额头上取了下来。

我:“那个…!学长…!总之…今天谢谢你…!那个…我…要先回家啦…!”

内海看着我慌张的样子,一脸憋笑:“嘛,是啊!明天记得把额头前的头发整理一下,再被我抓到给你记过!”

我看着内海,不知为何害羞地躲开了他的视线,匆忙地拿起地上的包,往校门口走去。走到一半,我突然想起什么,又走回到内海的旁边。

“那个…学长…我的名字…”

“美月对吧,刚才偷听到了。”

“诶…?!”

在他的注视下,夕阳悄悄爬上了我的脸颊。

“我…我可没同意…让学长直接叫我的名字!”说完,我便生着气逃也似地跑开了。

好讨厌!好讨厌!为什么心脏…跳得这么快啊…!


\cutlinef{今天回家的路上稍微绕了些远路}

\newpage

\newday{\food\sunny}

第二天,下课期间,我一个人躲在教室的角落里看着书,悄悄听着旁边女生们的八卦。



女生甲:“诶?你们听说了吗?隔壁班有个男生今天来学校被抓到书包里有烟!”

女生乙:“要我说现在才抓到也太慢了吧?我和他擦肩而过的时候都能闻到诶!尤其是他从厕所回来的时候!”

女生丙:“我也听说了,好像还是那个内海抓的。”

女生甲:“确定?!内海学长平常不会亲自出手吧?!平时不都是副学生会长在校门口查吗!我今天来的时候还是她!?”

女生丙:“听说内海学长像是专门在门口等着他一样,他一进校门就从不知道哪个草丛钻出来,把他给逮个正着!”

女生甲:“哇,他还是那么喜欢钻草丛啊…嘛,毕竟他一直在躲着佐藤さん。”

女生乙:“诶?他俩还在僵持吗?要我说佐藤さん早该换个人了。内海学长的意思都这么明显了,为什么还这样死缠烂打呢?”

女生丙:“佐藤学姐就是这样的人吧…会长的位置没竞争过内海,所以就想直接把会长本人拿下…是这个逻辑没错吧?”

我偷听到了什么关键的信息:原来内海学长是学生会长吗?!

女生甲:“但是也太钻牛角尖了吧?她那个级别的人物,追求者都能凑出一个班了!”

女生丙:“所以说啊,她唯一看得上的就只有比她高一级别的内海了吧?无解。”

女生甲:“有解吧!只要内海学长找个女朋友,佐藤应该就会死心了吧~”

女生乙:“我看也未必。这个节骨眼去当内海的女朋友…那不就是明摆着佐藤学姐对着干吗…之前多少追求内海的女生都是因为佐藤学姐在那杵着,就都停手了。反正要是我,我肯定不干。”

女生丙:“对啊…内海现在高三,佐藤现在高二,等下学期换届,那佐藤基本是稳定当会长的,到时候被佐藤当眼中钉…后面一年都没好日子过咯。”

几个女生聊完学生会的八卦,又开始聊起其他事情,一直到上课铃响起。

我的高中生活本该是如此枯燥才对,孤独地看书,像这样偷听别人八卦,上课偶尔被老师叫起来回答问题,要是数学课还大概率答不上来。午餐也是一个人找一个阴暗的角落吃饭,只希望不会有人来打扰。除了昨天和学生会长的小插曲外,这就是我最熟悉的日常了。或许今天唯一的不同就是父母又去出差了,而我今天又赖了床,所以没准备盒饭。于是当转眼间来到午餐时间,我便久违地去了食堂就餐。

\cutlinef\food

为了避开高峰期,我一般都是选择午休铃十分钟之后再去,那时候食堂的人数基本就减半了,也会空出很多单人的餐桌。

当我来到食堂挑好午餐,就开始找空位置。当我找到一个好位置,正要过去时,突然被一个熟悉又陌生的声音叫住。

“美月!”

当我正在回忆着是谁的声音时,我朝着声音的方向看了过去,果不其然,是内海学长。

内海朝我挥挥手,示意我过去。

虽然很想一个人待着,但是不知道为什么,我还是坐到了内海学长的对面。

我:“怎么…内海…学长(\japan{せんぱい})…找我…有事吗?”

内海:“呀,美月啊,帮学长个忙。你也不用说话,就坐在这跟我吃饭,就算帮忙了,好不好?”

我:“唔…可以…不过…”

内海好奇地瞪大眼睛:“不过什么?”

我突然有些害羞地说道:“不要直接叫我的名字…”

内海:“诶?那叫什么?美月さん?美月ちゃん?”

我:“我姓…雨宫(\japan{あめみや})…”

内海:“雨宫…念起来没有美月好听啊?”

我的脸上挂着生气:“那也不行…不要那样叫…不然我就走了…”

看到我假装起身的动作,内海连忙叫住我:“别别别,我真需要你的帮忙,雨宫ちゃん,就在这吃饭吧…”

我看到内海妥协后,居然得意地笑了。

我将侧发别到耳后便开始就餐,内海学长看着我笑了笑,也动起了筷子。

内海:“雨宫ちゃん,你头发有好好打理呢,看起来精神多了!”

我听了内海学长的话,回家的路上找了家理发店修了修前面的长发,最终成了不至于遮住眼睛的齐刘海。不过稍稍低下头也能用前发挡住别人的视线,自认为是机智的妥协。

我:“没…反正…不会有人在意的…”

内海:“我会在意啊~”

不知道内海学长说这话的意义,我只好害羞道:“你…毕竟是学生会长…”

内海:“诶?你知道了吗?明明昨天还问我是谁来着?”

我:“偶然…知道的…”

内海:“但是多亏你昨天的情报,我抓到了一个在学校抽烟的家伙。这还是要谢谢你呢~”

我:“没…没什么…”

我们就这样边吃边聊,午餐结束后,内海递给我一盒纯牛奶。

内海:“诺,犒劳。”

我道了声谢,接过内海的奖励,离午休结束还有一段时间,我们就这样坐着消化食物闲聊。

内海:“雨宫ちゃん,也是堂食派吗?”

我:“最近是…但不一直是…”

内海:“那我们是一类人呢。”

我摇摇头:“不…不是一类人…”

我仔细琢磨着学长的每一句提问,希望能够尽量与他保持距离,以免让自己卷入多余的事件之中。我本来是这么想的,可是或许从我坐在他面前开始,我就着了魔似的被学长的话牵着走。

内海:“诶,别那么果断嘛!既然雨宫ちゃん最近都是来食堂吃饭,那接下来几天也能像今天一样陪我吃饭吗?”

我:“不…不行…今天是为了让你改正称呼…才跟你吃饭的…”

内海看上去有些失望。

内海:“那可麻烦了…”

我还不知道内海到底为什么要叫我一起吃饭。

我:“但…但是如果学长告诉我原因的话…我就考虑一下…”

内海:“哦~说来也是呢,你肯定觉得莫名其妙吧!”

我点点头。

内海:“但是说起这事呢…怎么说呢…你知道我和副会的事吗?”

我点点头:“偶然…知道了…”

内海:“唔,那就好。那简单来说呢…就是最近被她发现我吃餐盒的草丛了…所以我要换个地方掩护自己。”

我一边回忆着昨天沾满落叶的学长、一边用这回忆想象着学长躲在草丛里吃餐盒的狼狈模样,偷偷地笑了笑。

我:“那…那为什么…要我一起吃饭呢?”

内海:“因为我觉得该跟她做个了断了…之前一直躲她已经让我有些烦躁了…”

我还是没明白:“跟我的关系是…?”

内海:“所以我希望你能装作我的女朋友。只需要装就行了!就像这样一起吃饭,其他的事情都不需要你做。”

我:“…诶?!”

内海学长一脸认真的看着我,看得出来他对这件事是认真的。

内海用手摸了摸后脑勺:“嘛,这件事毕竟也是我有求于你,所以就算你拒绝也无所谓,我再花时间找个合适的人选就行了。但是如果可以的话,还是希望雨宫ちゃん能考虑一下,每天同一时间在这张桌子上一起吃个饭就行。”

我不解:“唔…我是合适的人选吗?…为什么是我呢?…明明我在班级里…只是个透明人…”

内海托着下巴说道:“唔…应该说是命运吗?因为昨天仔细观察了雨宫ちゃん的眼睛,感觉非常漂亮,有一种不可思议的感觉。很干净?很清澈?”内海灵光一闪般,突然举起一根手指补充道,“啊,还有!虽然看得出来眼神一直想躲闪,不过好像最后还是会礼貌地看着我哦!那时就感觉这个女孩好有意思,或许就是「一见钟情」吧?”

我悄悄把头埋下去,不想让内海学长看到我此刻的表情。

什么啊…谁的眼睛都是那样的吧…这分明就是在搭讪吧!不会已经用这种方式骗过好多女生了吧?!但是…昨天学长的眼神才是很难不让人在意的那方吧…唔,感觉遇到陷阱了!一定不能掉进去…要给他一个漂亮的反击!!!

我赌气般拿着餐盘站了起来,俯视着内海学长说:“要是…要是只是装情侣的话…就不要来麻烦我了!”

我生着气从他的面前逃开,完全没反应过来自己为什么生气。也完全没有注意到,身后被吼蒙的内海学长,露出了多么有趣的表情。

\cutlinef\timepast

第三天午餐时间。

内海:“哟!雨宫ちゃん!”

我看到内海的招呼,径直地坐到他的对面。

内海:“看来雨宫ちゃん是同意了呢?”

我:“没…只是…看到这里没人坐而已…”

内海笑了笑,两人心照不宣地开始餐食。

我:“内海学长…要这样多久…”

内海:“可以的话…一直到下学期学生会选举结束?”

我:“唔…最近妈妈就会回来…到时候应该就要带餐盒了…”

内海:“没关系,没关系。麻烦你一整个学期我也不太好意思。所以到时候就…”

我:“我…!会带餐盒过来的…你看也有人…会来这里吃餐盒…”

内海被我打断了话,惊讶着看着我。

内海:“雨宫ちゃん…”

我:“别…别误会…只是想报答…学长前天帮我的忙…而且…我也没有一起吃饭的朋友…所以就正好…有个说话的伴…”

说完我的脸泛起了红晕。

内海放下餐具仰面感叹道:“朋友啊~雨宫ちゃん原来已经把我当朋友,真开心呐~”

我连忙狡辩:“现在还…还不是朋友…”

内海仰着面笑了笑,就像是已经猜到了我会这么说。

就这样,我和内海学长每天中午都相约着一起吃中饭,渐渐地,我也在学长的影响下,稍微变得开朗了一些。

我:“喂!草丛学长,你又在挑食!”

内海:“这个不好吃!又硬又没味道!”

我:“这是胡萝卜!不是硬!是脆!而且味道你要好好品尝才有啊!就跟米饭一样!”

大概就是开始共餐后没几天,内海学长因为没带够钱,只买得起一小碗连我看着都嫌少的面。我的便当还没吃一半,他就已经舔干净碗筷,不怀好意地死盯着我的便当。那时的我受不了他那无助的眼神,只好象征性地给他夹了一小块炸猪排。现在回想起来,大概就是那次开始,我就慢慢察觉到学长其实很少吃荤菜,我在早晨准备便当时,就会以防万一似地多备一些菜。再后来就演变成了,我吃不掉的菜总会用各种方式塞给他,就这样以某种方式实现了午餐的合理分配。而在某次喂食的时候,我也找到了学长最讨厌的食物——这也是为什么我总是隔三差五地给他准备炒胡萝卜。

内海:“雨宫さん可真是啰嗦啊!明明一个月前连舌头都捋不顺。”

我:“我…我哪有捋不顺?!”

内海用手指着我:“你看?!”

我用不至于发出响声的力度拍着桌子:“你…!”

内海:“我怎么了!胡萝卜你这么喜欢吃你全拿去吃吧!”

我娇嗔着站起来:“你不吃那明天就别想见到我!”

内海难以置信地看着我:“你…?!你居然威胁我?”

我:“是学长找我帮忙的吧?!既然你不需要帮忙的话,那我就不自讨没趣了~!”

“我…”内海突然间哑口无言,几秒的沉默后,像是瞬间变了个人般说道,“我觉得胡萝卜其实还是挺好吃的。”

内海说着就把我为他做的炒胡萝卜丝递进嘴里啃起来,我见他服软,就哼了一声坐了下来。

内海眼里突然放光:“嗯!这是真好吃诶!要不是雨宫做的,我真以为胡萝卜是这世界上最难吃的食物了!”

看着内海一副吃得津津有味的样子,我满意地笑了。一直到后来结婚和他提起这事,他才告诉我当时是在阴阳怪气我。

内海吃完后,惯例将一盒纯牛奶递给我。

“诶?又是牛奶?能不能换换口味?”我自然地接过牛奶打开口子,反倒是挑剔起来。

内海:“送你的还挑?”

我:“我给你带的菜可不会重样!”

内海摸着下巴:“唔,上周三的和这周二的都是肉沫茄子。”

我:“喂!这是我亲手做的诶,和你那从超市一买买一箱的能比吗?!”

内海:“好~雨宫さん说的都对~您想喝什么?小的下次给您买?”

我喝了一口说道:“香蕉牛奶吧。”

内海:“不还是牛奶?”

我拍拍桌子:“你没听到「香蕉」二字吗?”

内海:“行行行。我不如给你买串香蕉,你边吃香蕉边喝牛奶吧!”

我:“不是!学长你怎么这么抠啊!”

内海:“你这不无理取闹吗!”

我:“我看你才无理取闹呢!我陪你在这吃饭,我有任何好处吗!”

内海:“所以才给你牛奶啊!”

我憋红了脸:“我想要的又不是这个!”

内海被我的话吓到,当我意识到我自己说了什么的时候,气氛已经僵持了一段时间。

说起来我到底为什么会陪学长吃饭?

内海思考了片刻后说道:“那…交换联系方式吧?”

我:“…诶?怎么突然…”

内海:“不是你说的吗…”

我:“我…我也没说现在…”

内海:“那…不要?”

我:“唔…也没说不要…”

内海:“到底哪边?”

我慌张到最后还是妥协了:“那…那…请学长…把手伸过来…”

我紧张地用左手捏着学长的食指,右手在他的手心写下家里的固话号码,随后也将手心摊在他面前,他左手鸿毛般托着我的手背,我盯了眼学长纤长的手指,又偷偷瞄了眼学长,他有些不好意思的神情,通过笔尖一点点地刻在心头,让人直痒痒。

写罢,他的左手掌心仍停留在我的手背上,像是下定什么决心似地说着:“那…这周六晚上七点,我会给你打电话。”

我:“唔…好…”

跟刚才嚣张跋扈的姿态完全不同,此刻的我就像一只闯祸了的小猫一样不敢作声。

我痴痴地看着手心里的数码,不断地默念着,希望能将它牢牢记在心里,嘴角却早已不自觉地微微上扬。

\cutlinef{学长给我留下的也是固话号码}

\newday{{\harfmoonr\night}}

周六晚,我在一个人的家里等待着电话铃声的响起。

当七点的钟声响起,一通电话也如约而至。

我立马跑过去接起电话:“喂!?”

电话的那头:“秒接啊,看来是雨宫家吧。”

我听到熟悉的声音,焦躁的心突然安静下来。

我平复了一下心情,以防对方察觉:“所以要说什么…?”

电话的那头:“我其实也没想好…我平时也不给别人打电话。”

我:“我…我也是…”

明明已经拉近距离的两个人,隔着电话却都腼腆了起来,惹出一阵沉默声。



我\&电话的那头:“那个…”

电话的两边都传来笑声。

我:“你先说…”

电话的那头:“要不…明天一起去哪里玩吧…?”

我偷笑着:“诶?这是邀请我去约会吗?”

电话的那头:“按照约定,算是假情侣的一环。”

我故作轻松道:“是假的就不去了~”

电话的那头沉默了。

我听着那边的沉默声,心脏跳到了嗓子眼,突然开始纠结要不要服软跟他去…

当我正要说什么,电话的那头用冷静沉稳的声音说着:“是认真的约会。去吗?”

我难掩心中的喜悦跳了起来:“去!”

学长告诉我他来定位置,然后明早九点在离我家一公里的车站碰面。我由于没有什么约会经验便也同意了。与其说是约会经验,倒不如说连外出游玩的经验都少之又少。后续我们聊着一些有的没的,不知不觉一小时就从指缝间溜走,直到那边传来了女性喊他名字的声音,他才匆匆挂断电话。应该是他的母亲吧?幸好今晚我爸妈还没回来,稍微可以放纵一些。

我们所住的城市有一个出名的水族馆,爸妈有事没事就带我去那玩,听他们提起他们当时就是在那里相恋的,所以就像是带着我圣地巡游似的。但总感觉他们只是想去那里回顾青春,有我没我终归一个样。所以我慢慢地也就厌烦了水族馆,但唯独每当来到最深处的水母馆时,总会被水母的样子给吸引。但是不管怎样,如果让我选,我不会把地点约在水族馆。所以学长会带我去哪呢?就这样想着,我在床上辗转反侧,不一会思绪又开始纠结起明天的造型。

“齐刘海要不要卷一下?要穿裙子吗…还是裤子…先看一下温度!唔,不算太冷,好像可以穿裙子?晚上会降温…应该不会玩到太晚吧?那就裙子吧!鞋子…应该也不会要走太远吧…?就穿帆布鞋吧!…”我抱着床单辗转反侧,不知不觉间便安稳地睡着了,梦里仿佛也出现了和他约会的场景,无法分清是睡前的幻想还是梦中的现实。

\cutlinef\zzz

次日清晨,我站在落地镜前,若是平日上学定是随意抓一下头发,将制服整理平整就了事了,今天却格外地在意着自己的气质。我在镜前一边回忆起睡前的计划,一边摆弄着姿势反复斟酌,心情激动又忐忑,担心自己的仪容不符合学长的品味或是走在一起时不体面。时间就这样在少女怀春之中不经意地溜走,当注意到就快要赶不及时,终于决定好了穿着冲出了家门。

今天正如天气预报所说的,是晴朗的白天,阳光撒在身上暖洋洋地恰到好处。春末的气息仍飘荡在空中,不冷不热不湿不燥,不时吹来伴有青草气息的清风,仿佛能让人轻易地放下心中的戒备,或许称为最佳的约会时节也不为过吧。

我一路小跑着往约好的车站赶,秀发与裙边飞舞着卷起凉意,将内心的期待展露无余。当快要到车站时,我望见内海学长已经站在了车站的站牌边。

我喘着气向着学长打招呼:“抱歉…学长!我来晚了…”



内海挠了挠脸颊:“没…我也才到。”

说着内海将一盒香蕉牛奶递了过来:“看你跑这么多汗,渴了吧?”

我笑着接过香蕉牛奶:“看到你就不渴啦~”

无心的一句话将两人的耳根羞得通红。

我慌得连忙摆手:“我…我刚才…”

\newday{\windy\cloudy}

内海也连忙将手插进裤兜,用笨拙的装酷掩饰着情绪,视线看向远处的轨道像是抓到什么救命稻草般说道:“那…那个车来了,上车吧…!”

%\def\cla{white}
%\def\clb{brownl}
%\def\clc{darkbrown}
%\def\cld{rice}
\dressdisplay{white}{brownl}{darkbrown}{rice}

我把香蕉牛奶放进随身的\colorfy{rice}{米色托特包}里,跟在学长的身后上了车。



市区的电车总是人流涌动、挤满着各式各样的人,我们在靠边的车厢上车,进门后在一块没人的拉手处彼此挨着。趁此时机,我才有功夫打量起学长。他穿着白棕拼色的毛绒翻领外套,里面是白色的\colorfy{white}{短袖衫},下身是简单的\colorfy{gray}{直筒休闲裤},再往下便是黑白配色的\colorfys{black}{\color{white}{板鞋}},脚踝裸露在浅口的白色船袜外面,与学长本就清秀的脸配合得恰到好处。头发看起来也比在学校里干净,发丝在阳光下反着光,能看出来才刚洗过。清爽的气质与在学校里总是发间夹杂着绿叶的模样比起来简直判若两人,但望向他那双深棕色的瞳孔时,我仍能清楚地辨认出这就是学长。

悄悄贴近学长,能闻到T恤上传来一阵好闻的、只属于学长的味道。当意识到学长也在认真对待这场约会后,躁动不安的心稍微平静了下来。



内海拍了拍我的肩膀:“雨…雨宫さん?”

被内海一提醒,才发现自己在不知不觉间被香味吸引,已经靠在了学长身上。

我赶紧起身:“啊…!抱歉学长…那个…有点累…!”

内海:“啊…这样啊…那…嗯…也没办法嘛…!”

学长慌张的样子也和我一样呢。

我低着头说道:“其实是因为…学长身上有股很好闻的味道,所以不自觉就靠过来了…”

“诶?!我怎么闻不到…?”内海抬起手臂闻了闻,“哦…你说的应该是樟脑的味道吧?最近妈妈在衣柜里放来着。”

我又凑上前闻了闻,好像确实也有樟脑的气味。

内海:“不过我觉得还是雨宫さん的味道更好闻哦~连衣裙也很适合你~”

我腼腆地回道:“谢谢学长夸奖~嘿嘿~稍微打扮了一下。”



我出门前花了不少时间洗头吹干,卷了卷发尾,又将刘海卷了卷,将一边的刘海别到耳后,梳成露出半个额头的斜刘海。镜子前挑了半天,最终选择了一件白色肩带渐变到水色裙摆的\colorffy{left color=white,right color = lblue}{及膝雪纺连衣裙},无袖袖口与娃娃领都带着花边,下半身是三层荷叶边的蛋糕裙设计,束着腰还带着些闪粉。怕着凉肩部又披着米白色的\colorfy{lwhite}{中袖针织开衫},鞋子则是浅蓝色的\colorfys{lblue}{休闲帆布鞋},里面穿着随处可见的白色短袜。这些都是去年生日父母帮我挑的,本来只在生日那天穿过一次,不曾想过有一天它们会再派上用场。

%\dressa{whitel}{lblue}

内海:“但是,晚上会降温吧?听说是春季结束前的最后一次降温,这样穿晚上不会冷吗?”

我摇摇头:“至少现在不冷~而且要约会的话,还是要穿得好看些吧。真要降温早点回去就是啦~”

内海温柔地笑着:“雨宫さん和在学校的样子完全不一样呢~”

我回敬了一个浅浅的笑容:“学长也是~”

只是这样简单的奉承,我们便傻乎乎地笑起来。日光洒进车厢,照在那情窦初开的脸庞上。

电车到站了,我与学长下了车,但其实我还不知道我们的目的地。

我:“学长,我们要去哪?”

内海:“跟着我就知道啦!”

学长在前方带路,我跟在学长身后保持着二十公分的距离。

我突然意识到什么:“不会…是水族馆吧?”

内海一听回过头笑道:“诶?雨宫さん怎么知道?”

我:“真无聊诶!你是不是随便在网上搜的啊,一搜约会肯定开头就是水族馆!而且这个地方…这个方向…不就是大家平常都会去的那个吗?”

“啊…?是嘛~我倒是挺想带雨宫さん来这玩的!”内海摸摸后脑勺,“不过我也是第一次和女孩约会,所以不知道女孩会喜欢去哪里…”

我一听,心底突然开出一朵花~

我坏笑道:“诶?原来内海学长也是第一次约会嘛~?完全想不到诶?”

内海有些苦恼地闭上眼:“嘛…毕竟学生会还是很忙的…这方面我平时也不关心…所以要是雨宫さん有更好的建议的话,下次就让你挑地方吧~”

我哼着笑道:“内海学长在这方面也是个新手呢~”

内海:“什么方面…?约会?”

我:“「女孩的心思」方面。”

内海:“什么意思啊?我倒也没你想得那么不懂吧!联系方式不也是我先说的要换嘛!”

内海不甘被我当成是恋爱新手,开始在自己的回忆里搜索着反驳的证据。他本想把另一件事情也说出来,但怕毁了今天的氛围,便选择了暂时不讲。

“好好好~水族馆~$\eighthnote$水族馆~$\eighthnote$”

我不知怎的,开始对本来没什么兴趣的水族馆充满了期待,脚步也不自觉地变得轻快起来。

\cutlinef{水族馆也不赖嘛}

“看!学长!那是鲨鱼!那是鳐鱼!趴在那好呆!”我一边拍着学长的手臂一边指着头顶飘过的鱼兴奋地叫着。

“看!学长!白鲸!那是小丑鱼!”学长被我拉着到处跑。

“看!学长!好多叫不出名字的小鱼!学长知道叫什么吗!”

内海被眼前活蹦乱跳的大眼睛注视着,有些不好意思地摇摇头。

即使不知道名字,各式各样的鱼也吸引着我的视线。明明以为是看腻了的风景,现在却变得异常鲜活。

逛了不久,我们便来到了水族馆的最深处。

“看!学长!是水母!哇!透明的样子好漂亮!”

这就是这个水族馆最后的一个景点了。数十种水母分散在各个缸体中,海蜇水母、水晶水母、火焰水母、狮鬃水母……形态各异,色彩万千。而最受欢迎的海月水母,集中在正中央的巨大缸体中,横跨过整个水族馆。数以万计的海月水母自由自在地游荡在其中,在周期更迭的光源照射下,散射着如彩虹般的光芒。

如此缥缈,如此梦幻。

我看着水母入迷,而学长则跟在我身后一副生无可恋的样子。

看着一只只浮动着的水母,我的视线泛起一层朦雾,将自己的思绪带回过去。曾经与父母来到此处时,只有我会呆呆地望着这些生物发呆,自己孤独的身影与这些水母一起被定格在那一瞬间。

我这才回忆起来:“学长…我其实很讨厌水母…。”

内海刚从被拽来拽去的氛围中回过神来,还不明白发生了什么:“诶?水母?为什么?”

“嗯…因为感觉自己和水母很像呢…”我看着眼前漂流着的水母,稍显失落地耷拉下眼皮:“学长知道吗,水母(\japan{クラゲ})如果写作海月(\japan{くらげ})的话,也是可以读作\japan{みつき}(海月)的,和我的名字一样呢。”我在玻璃幕墙上随意地比划着「海月」二字,“所以美月呢,就像是这些水母一样,只能这样孤独地游来游去,没人会注意到、也没有人会在意最后会到哪去…而且你看,水母是透明的,就跟我一样…”

内海看着我的手指写完自己名字的最后一划,随心地开起玩笑:“而且靠近了会还会被蜇呢!”

我将手指收到半空中,瘪了瘪嘴:“我…可没蜇过学长…”

内海看着我失落的神情,笑着说道:“但是我觉得雨宫说得不对哦。”

“哪里不对呢?”我没有回头,目光仍然在那些水母上,仿佛在坚守着自己的想法。

内海也别过头去,同一样的视线看向在水中漂浮着的水母:“你看,即使是你觉得没人在意的水母,也有像我们、还有很多游客来观赏着。所以即使雨宫是水母,一定也会有注视着你的人存在吧。”

“就算注视也只是讨厌的目光吧?就像我现在看着他们,讨厌的要命…心底里大声地咒骂着「不要再随波逐流了!快去找到自己喜欢的东西吧!」。”我自嘲着,“就像是照着镜子,只有咒骂着水母,我才能听到自己内心最深处的呐喊…但是离开这里后,一切都没有改变,一切都无法改变…我仍是这样随波逐流着活到现在,我讨厌这样的自己…”

明明这些话从来没对别人说过,但不知为何,如果是内海学长的话,内心稍微愿意放下一些戒备。

“如果这真的称得上「讨厌」的话,那雨宫さん又为什么一直用着充满期待的目光注视她们呢?”

…我的目光?

我有些惊奇地看向学长,他的目光仍聚焦在那些水母身上。

“雨宫さん其实很喜欢水母吧,刚才就那么兴奋地一路跑过来!难道雨宫さん会对讨厌的东西有所期待吗?我不这么认为。「讨厌」的外表下,是发自内心对她们的「喜欢」才对吧?”内海微微低下头,像是在遣词造句又或像是触景生情般,最终转过头来视线锁定到我的眼眸,“而且你说水母是透明的,那换个角度讲,水母的心也是透明的。所以我觉得或许更确切地说,雨宫さん的心才是像水母一样,透明、清澈、又像这样发着美轮美奂的光。”

那些从来没有人对我说过的话悄悄地扎根在我的心里。

我低下头:“但是,就算这样…我也只是只不会发光的水母…”

“谁说的!?”内海的质问将我的目光拉回到眼前的男生身上,“在我眼里雨宫さん可是发着耀眼的光哦!雨宫さん只是还没发现自己身上发着光呢!…看!雨宫さん的眼睛现在就闪着光!”

果然学长的眼睛就像深海一样,又幽静又深情,真令人着迷呐…在那深海之中,仿佛能寻见一只飘荡着的水母,一只不会发光的水母…原来学长就是那个注视着我的人…而此刻的我,究竟用着什么样的目光注视着学长呢…?

那目光中的光逐渐凝聚成滴。

…笨蛋学长…那才不是光吧…!

“雨宫…さん?”

我:“那个…呜…学长…能借你的衣服当纸巾吗…?”

内海看着我温柔地笑着:“可以哦~”

我扑在内海的怀里开始啜泣起来:“从来…呜…从来没有人…跟我说过这些…”

内海安慰着抚摸我的头:“那雨宫さん从今天开始,一定会听到更多这样的话的。”

我将双手贴在学长的上腹,眼泪肆意地洒在学长的T恤上:“…呜呜…都是…学长的错…”

“嗯…对不起呢~”

“不要说对不起…呜…”

“嗯…抱歉~”

我轻轻地锤着学长,却被学长的手轻轻接住。想着挣脱,学长却没有松手。于是我便将手摊开,任由他抓在手心。

我睁开眼低头看着被学长抓住的手:“学长…”

内海轻柔地问道:“可以吗…?”

内海用拇指轻轻抚摸着我的手背。

我点点头,渐渐停下了啜泣,紧紧地回握住这只滚烫的手。

内海拍拍我的头,我才从内海的身上离开。

内海:“好啦~后面应该还有露天海豚,去看看吧?”

我擦了擦眼泪,点了点头。

在路过周边区时,内海撩了撩我的刘海:“你看,你这边的头发都掉下来啦!”

应该是刚才在学长的怀里蹭来蹭去的缘故吧,本来在耳朵后面夹得好好的刘海掉了下来,变回了原来的齐刘海。内海便顺势为我买了一个淡蓝色的水母发夹,将我的半边刘海别好,帮我整了整刚才乱掉的头发。发夹上的荧光在灯光下闪着光。

“嗯!水母ちゃん!”

“不准乱起外号!”



内海笑着转过身去,将右手掌摊开,像是在邀请着什么。我心领神会般,羞涩又明快地将手递了上去,仿佛一切本该如此。



\cutlinef{那是我们第一次牵手}




“哇!学长!是海豚!好可爱!”

我坐在露天表演的观众席上,一边用手指着水池中央的海豚,一边朝着学长兴奋地叫喊着。等到表演结束,我便拽着不情不愿的学长来到水池边摸海豚。

“哇!学长!摸起来滑滑的!”

“看!学长!海豚在舔我的手!好痒~!”

“学长不来试试吗!?”

“诶?学长怎么这么胆小!海豚明明这么可爱!”

看着女孩天真期待的眼神,内海只能无奈地靠近身子,缓缓伸出手摸了摸海豚。

“学长!很好玩吧!”

内海回想着某人明明才大哭一场,现在又变得元气精神的模样,也忍不住眯着眼笑了起来。

我们终于从水族馆出来,内海牵着我的手,指着前方的大商城问道:“雨宫さん,接下来要去吃点什么吗?”

我笑着点点头:“嗯~学长带够钱了吗?”

内海:“嗯?要我请吗?”

我摇摇头:“我的意思是要是学长没钱的话我可以请客~”

内海:“啊~那不用担心。”

我们说说笑笑着走向大商城。在旁人的眼里,我们一定是一对恩爱的情侣吧。我这样想着。

“学长!那个毛绒玩具比你还大诶!”

“学长!这个冰激凌看着好好吃!好像是青苹果味诶!学长要不要一起试试?……哇!好好吃!”

“学长!这个耳环好好看!我要不要去打个耳洞呢?……诶?!学校禁止吗?”

“学长!这个毛绒发圈好可爱!像只小兔子一样!诶?买来试试?不会吧?学长喜欢双马尾吗?嘿嘿~那就没办法了呢~今天的双马尾是给学长的优待哦~诶?扎得再低一点吗?”

“学长!水母玩偶!看!这样抱着就是两只美月(\japan{クラゲ})了~嘿嘿~”

“学长!这个手帕有水母图案诶?!而且第二块半价!我们一起买吧!…对哦,这里有好多带着水母的周边,不愧是水族馆旁边的商城啊!”

“诶?学长喜欢制服鞋吗?但是学校的制服鞋是统一的嘛,其他款式的又不能在学校穿!”

“哇!学长不要再逛女区啦!要看到奇怪的东西啦!走啦!去看男生的衣服啦!”

“学长!这件外套很适合你诶!不适合吗?试试嘛!明明很帅!”

“学长!大头贴!要一起拍吗!诶?!为什么不嘛!”

我们就这样从一楼逛到顶楼,每一层都瞎溜达了一圈,随心所欲边走边聊。明明是被父母带来过好多次的商城,看着司空见惯的货品,说着毫无目的的对话,却也变得别有风味。托特包里的东西也越来越多,不知什么时候开始袋子就挂在了学长手上。

我们从商城吃完晚餐出来,此时的温度正如天气预报所说得一样变得异常寒冷。

我不禁打了个哆嗦:“呜…好冷…”

内海:“不是说了要降温嘛!”

我狠狠捏了捏内海的手:“那我还说要降温前回去呐!”

内海笑笑,把自己的外套脱下披在我身上,自己则拿出刚才一起买的夏季外套披在身上。

我诧异地看着仿佛是活在另一个季节的学长:“学长…不冷吗?”

内海:“毕竟我是阴虚体质嘛~穿这件外套来就是以免某人穿少了,白天的时候我可是闷出来好多汗!所以现在只是把外套披到了更合适的人身上~”

我看着学长的T恤,似乎还能找到一丝汗渍,也才明白紧握着的手为何总是如此温暖。我怯怯地裹紧学长的外套。

内海抓着我的手抬起来:“所以雨宫さん的手握着很舒服哦!雨宫さん的手冰冰凉凉的~”

我:“哼~那学长可要让它好好暖起来哦~”

内海笑笑:“了解~”



内海:“那…接下来就该回去了吧?明天还要上学呢~”















\cutlinef{学长的外套暖烘烘的,还有学长的味道}

\newday{\windy\cloud\harfmoonr\night}


内海牵着我来到车站,我们就这样挤在长椅的一边等车,我的身体几乎黏在了内海身上,可我们毫不在意。

我吩咐内海把托特包里的水母玩偶拿出来:“嘻嘻~水母~$\eighthnote$水母~$\eighthnote$”

我在内海面前摆弄着玩偶模仿着水母游泳的样子。

内海:“吼~大水母在玩小水母!”

我:“谁是大水母哦!”

内海笑笑:“会是谁呢?要不以后还是叫你水母ちゃん吧~就像你叫我草丛学长一样~”

我憋红了脸:“都说了不要乱起外号!而且那是因为学长在学校身上总是有杂草和落叶!”

内海:“那雨宫さん不也很像水母吗!而且在水族馆的时候还说自己是只水母呢!你看扎了双马尾后也很像是水母的触手呢~”

内海说着用手指绕着我的双马尾说道,我没有躲闪,反而是配合着甩了甩马尾。

“哇!原来学长让我扎低马尾就是为了这个吗!”似是埋怨,内心却雀跃着,“哼!那也不要叫水母(\japan{クラゲ}),要叫海月(\japan{みつき})吧!”

内海坏笑着:“好~美月(\japan{みつき})~”

我还没回过神来,内海就直接叫出了我的名字。

“是海月(\japan{みつき})…不是美月(\japan{みつき})!…哇…不对!好像是一样的啊!…呜…”

“反正就是みつき,对吧~”

我羞着脸,拿着水母玩偶砸向内海:“你…!”

内海:“疼疼!”

内海假装被打疼的样子。

我吓得将水母玩偶收回来,用玩偶捂着自己的嘴巴:“没…没事吧…学长…”

内海一副毫发无伤的样子:“哈哈~美月真可爱呐~”

我生着闷气鼓起腮:“呜…只有今天…只有今天允许学长叫我美月!”

内海:“好好~美月~”

我:“嗯…学长!”

不知为何,心底里倒是期待起学长直呼我的名字。真是可恶啊!居然这么容易就被学长得逞了!学长似乎总能看穿我扭捏的内心。

电车快到了,我便把水母玩偶收了起来。

我突然想到什么笑了起来:“说起来…和学长第一次见面的时候,学长就是这么叫我的呢…”

“但是第二天就被美月禁止了呢~”

“因为学长那时候就像个笨蛋一样,随便叫着陌生人的名字!”

“那美月才更像个笨蛋吧,社恐到连话都说不清楚~”

“那…今天就是两个笨蛋的约会呢~”

听完内海也笑了,两只手仍紧握在一起。

内海:“所以美月真的很像水母呢~初次见面的时候完全想不到,美月是这样可爱的女孩~”

我:“我才是,完全没想到学长这么坏~”

内海:“那我就当你是在夸我咯~”

我:“才没有在夸你!”

内海笑笑。

内海突然意识到什么:“说起来,这就好像是水母用触手在保护自己一样呢~”

我:“什么?”

内海转过头来望穿我的眼眸:“美月的傲娇~”

我:“我…我才没有傲娇!”

说着,我掐了掐学长的手。

内海开始装模作样将手背伸得老高:“哇!我被水母蜇到啦!”

于是我变本加厉地用另一只手掐着内海完全没有防备的腰,但是内海却像是被挠痒痒般笑了起来,看这招对他没什么效果,我只好作罢,生着闷气停下了无用的反击。

内海回过气来,认认真真地做好,像是在酝酿着完结的情绪,不一会便微笑着叹道:“真不希望结束呢,今晚!”

学长说着我内心的想法。今天对我来说也一定是特殊的一天吧。第一次与男生约会,第一次与男生逛街,第一次牵男生的手,准确地说是他牵起我的手吧!?第一次与男生坐的这么近,第一次感觉到心跳不属于自己…好多第一次都塞进了这短短的一天里,让人应接不暇。以至于回过神来的时候,所有的第一次都已成了过去式。这到底是一种什么感觉呢?!

内海突然侧着头看着我:“美月也喜欢的吧~?”

我受到惊吓般回望着学长的眼睛:“喜欢…什么?”

喜欢水族馆?喜欢今晚?还是喜欢…?

我的内心里开始期待着学长的下一句话。

微妙的氛围不允许内海再说些什么,他清秀的唇却在情愫地驱使下微微张开。

我盯着内海的眼睛,突然某种奇怪的氛围将我也慢慢地吸引过去,就仿佛坠入深海。我又一次在深海中发现了飘荡着的自己,随着学长微咪双眼,我也本能般地照了做——

哐当~哐当~列车已到站。

再看时,两人已不知何时变回了正坐的姿势,仿佛无事发生。

“哈哈…电车到了哈!”不知道该露出什么表情,万般无奈之下,内海选择了一个似笑非笑的诡异表情。

“嗯…很准点呢!”很不巧,那也是我的表情。

我看向内海,内海也看向我,那表情好似鬼脸,逗得我们又痴痴地笑了起来。

真是两个笨蛋的约会!

\newday{\love\windy\cloud\harfmoonr\night}

在列车上,我回忆着刚才的遗憾,反而变得不知如何开口。我看向内海学长时,他似乎也有着同样的神情。他思考时的侧颜也很让人着迷,但我一定会在他将头转过来前将视线移开。看!我果真提前移开了!可是学长却因此偷笑着。看来还是不够熟练呐!直到学长牵起手带我回家,我才发现什么似的问道:“诶?学长的家也在这个方向吗?”

	“说来也是呢,美月来的时候也是从这边过来的!”

	“对哦,因为学长是先到的,所以我完全没察觉呢~”

	“既然这样,那就先把美月送到家门口吧~”

我点点头,捏了捏学长的手:“正好也想多握着学长一会呢~”

这条习以为常的回家路上,多了一双紧握着的手。前几天才坏的路灯如心跳般闪烁着,将连接着的一对影子不断地拉长,邻里的宁静之中某种情绪正在心中膨胀着。那是自下车站时起,就在心中酝酿着的情绪,随着回家路上的每个步伐,逐渐变得清晰、变得无法忽视。

到家门口之后,学长就要离开了,今天就要结束了。可是我们仍然不是情侣…这一事实让我感到难过。不是情侣的约会…所以才称得上笨蛋的约会。松开这只手后,一切都会回到原样吧。学长还是学生会长,我还是那个班级里的透明人。我和学长的关系也只是一起吃中饭的合作关系,一切都没有改变。原来,这才是我不想让这一天结束的原因啊…

不要…

要是这样的话…为什么要答应来约会啊…空荡荡的手掌中第一次感受到了能直通心灵的温暖…我手中抓住的,或许就是传说中如泡沫般的幸福吧…?

“诶?这就是美月的家吗~是一户建啊!那美月岂不是个千金小姐!?”

“说起来美学家离我家还蛮近的诶~以前居然从没偶遇过!不过之前就算遇到也认不出来吧!啊哈哈~”

“那…我就送美月到这了?接下来我就要往那边走了…”

“哦对,我的外套!诺,还有你的包!”

“嗨呀,怎么哭了啊…?明天学校不就能见到面了吗?今天好好休息吧…!难得逛了一整天!”

“好啦,别哭啦~不然路人以为我在做什么坏事呢…”

“那么,明天见!美月!”

内海用食指指背轻轻擦了擦我的眼角,捏了捏我的脸颊后,便松开我的手转过身去,仿佛一切正在消失。

\highlight{“不要…!”}

我丢下手中的袋子,上前抓住内海的手,内海惊讶地又转回了身…我踮起脚尖,做出了自己都完全没法理解的行为…

顷刻间,内海的眼里只容得下一抹清凉的青风,神情逐渐从诧异变得柔情,最后也轻轻地闭上眼回敬了那位少女…

那是一股能让人融化却有转瞬即逝的温暖…随着那股温暖彻底消失,我双手仍搭在内海的肩上。两双清澈的瞳孔从未如此贴近过,可我的眼泪却怎么也停不住:“这…呜…这才是…水母(美月)的蜇刺…请学长…呜…好好记住…!”

糟透了,在这种时候哭了…但是终于把憋在心底的情绪传达出去了。

不想看到这个笨蛋,也不想让他看到像笨蛋一样的我,于是我抛下学长逃走了。当我洗完澡后,想起什么再出门时,发现我的托特包被安稳地放在了大门口。

我一边收拾着今天一起买的东西,回忆着今天的约会。

%
%憋了一天的情感,是否传达给他了呢?



%我想要逃避现实般地抛下内海。内海清楚地知道眼前正在发生什么,却没有因为雨宫的主动而变得开心,眼神反而变得惆怅了起来。内海呆呆地望着眼前落下的托特包,自嘲般地笑着,将袋子放到雨宫家门口。

%\cutlinef{}



“水母~$\eighthnote$发圈~$\eighthnote$耳钉~$\eighthnote$不过还不能戴呐呢…”

翻着翻着,突然发现了某个东西——

“诶?!”

在托特包的最底部,我发现了学长一开始就给我的那盒香蕉牛奶。我这才想起来约会期间完全没机会喝。于是我便插入吸管,一口一口地喝着。同时心里浮现出了刚才自己不理智的行为。

“哇!我怎么会做那种事情!学长会不会因为这件事讨厌我啊!明天该不该跟他道歉啊…?”

“明明还不是情侣…我为什么要这么做啊!那…可是我的初吻啊…!不过说起来应该也是学长的初吻吧?!”

“唔…那应该抵消了吧?!嗯!应该是这样的!哼哼~或许还是我占了学长便宜呢~!”

我躺在沙发上像小孩一样摆着腿,手心还在擅自回味着残留的温度,脸颊不自觉地发烫——这难道就是恋爱吗?!嘿嘿嘿~

“真想明天快点到来哇~”

香蕉牛奶很快就见底了。

\cutlinef{好甜!}

\newday{\rainy}

第二天,我戴着学长送我的发夹来到学校。如往常般的课间,内心的期待却非同寻常。

女生甲:“喂,你们听说了吗?昨天那个内海在水族馆那边和一个女生约会诶!”

女生乙:“嗨呀,这种重磅消息现在哪还有人不知道啊!”

女生丙:“内海学长果然还是找了个女朋友啊!佐藤さん现在应该死心了吧?”

女生甲:“最好这样吧!说到底这两人一直僵持着,把整个学校的氛围都搞得怪怪的。两个人都有问题!”

女生丙:“说起来,内海的女朋友长什么样?几年级的?”

女生甲:“唔,听消息来源说,两人穿得便服,完全看不出是几年级的。而且是他完全不认识的学生,或许就是我们高一的?”

女生乙:“哇,内海学长对着啥都不懂的高一生下手…真有心机啊!但是如果说是完全不认识,或许也可能是其他学校的学生?像是青梅竹马之类的?”

女生丙:“唔,也确实有可能啊!”

女生乙:“不过我还听说,那个女生看着普普通通的,不管是衣服还是发型什么的。难道内海喜欢的是那种「普通」类型吗?这或许就是内海看不上佐藤学姐的原因?”

%女生丙:“哇!那内海学长也太没追求了吧!”

%女生乙:“没办法的吧,毕竟连学校第一的女生都看不上,那相比之下谁都变成「普通」了吧!”

我尽量无视着三人令人不悦的对话,一直到午休的铃声响起。

女生丁:“雨宫さん!外面有人找你!”

我惊讶地望向门外。平常除了老师聊成绩的时候,都不会有人找我。快要到见学长的时间了,希望不会花太长时间。

我怀着忐忑的心情,走到教室外,最先映入眼帘的是一袭乌黑长发,那长发的主人此时正亭亭玉立地朝楼外看着那乌云密布的天空,听见脚步声便像是自言自语道:“快要下雨了呢。”说完转过身来,与背影不同,她有着非常强势的眼神。

女生:“是雨宫ちゃん吧,能和我聊会天吗?”

我:“请问…你是?”

女生:“嗨呀,居然连我都不认识吗?我可是每天都在学校门口巡逻的哦?”

我:“唔…是佐藤副会长吗?”

佐藤点点头。因为我进学校都是低着头,完全不会留意周围的环境,所以哪怕是每天都在门口的老师也完全没有印象。

我:“但是…我等下还有点事…”

佐藤笑笑:“要是你说的是内海会长那边的事的话,不用着急,我已经跟他打过招呼了。”

我似乎察觉到了佐藤来找我的目的。

佐藤:“那么…一起走走吧?”

佐藤领着我往学校里无人的角落走,当周围的学生显著变少后,佐藤便边走边说道:“我听说…昨天内海在和某个女生约会呢?”

我:“好像…是吧…”

佐藤冷笑一声:“雨宫ちゃん不要紧张,我知道是你。毕竟那个传闻就是我托人传出去的,我还特意帮你隐藏了身份呢!嘛,我也没必要和你撕破脸对吧~”

我:“你怎么知道是我…明明没几个人认识我…”

佐藤:“那个天天和内海会长在食堂吃午餐的女生——可是学生会的名人哦~所以看到你们照片的时候,我一眼就认出来了!”

我:“你想说什么…”

佐藤停了下来,转过身撩了撩长发,盯着我,看来这里就是散步的终点。

佐藤:“我是来跟你做交易的。”

我:“什…什么交易…”

佐藤:“离内海会长远点。”

佐藤的话让我的心脏停了一瞬。

我:“为…为什么…如果我拒绝呢…!”

佐藤大笑着将双手交叉在胸前:“看来你还有点天真呢。你还记得内海抓的那个抽烟的男生吧。要是我把是谁泄密的这件事告诉那个男生呢?”

我:“那你就说吧…反正他抽烟…迟早会暴露的…”

佐藤:“好~那如果,我们在你的书包里也发现了烟呢?”

我:“我…我又不抽…”

佐藤又凑近了身子威胁道:“那大家会是信你…还是信我呢?”

我:“要是只是…这种程度的威胁的话…还是别费力气了吧…”

我尝试着强硬起来。就算被查到带烟又如何,轻则批评教育,重则停学处分,不痛不痒!

佐藤突然歪着头凑近我的耳朵,将声音压低一档冷冰冰地说道:“那要是内海的书包里被发现什么呢?大家或许不会相信你一个阴森的女生会抽烟,但是那个人就不好说了吧?学生会可是有很多人可以帮我做伪证哦~”

学长现在还正好是高三,要是停学处分被记录下来的话…我突然开始有些慌张。

佐藤接着抬起身子恢复趾高气昂的语调地说道:“学生会长…在校内抽烟…这应该会记过吧?内海会长的成绩也不错,要是因此影响了他上重点大学…是不是不太好呢?”

我攥紧拳头,努力平息着心中的愤懑。

我:“学长不会抽烟的…”

“哇哦,你们的关系原来好到不用叫名字了吗?”佐藤假装吃惊,随后又冷笑着说道,“他有没有抽烟不取决于他…”

佐藤上前用手指按在我的胸口说道:“取决于你…”

佐藤看我一副惊恐万分的表情,开始奸笑起来:“安心~我不会让你白帮忙的。你大可以告诉我你喜欢上了内海哪一点,相貌?成绩?情商?还是那方面?无论哪个,我都能帮你在学校里找到一个比他更出色的!”

我低着头:“佐藤副会…那你又为什么要纠缠学长…?如果像你说的有那么多比他好的男生的话…”

佐藤大笑:“不缺比他好的男生~但是学生会长只有他一个!说回来,内海会长的追求者可不少,我可看过你的档案资料,成绩、社交圈、体育,你有哪个值得他费心?难道是样貌?难道是你的家庭背景?那学校可不缺比你能吸引男生的!或许他找你也只不过是一时兴起,拿你练手呢!过几天你就会被他踹到一边!”

我摇着头:“不…不可能!”

佐藤又换了个古怪的语气:“真可怜呢…多么纯情的孩子啊…真不想拆散你们~”

我回了她一个凶狠的眼神:“学生会长的身份…对你就这么重要吗…?”

佐藤咯咯地笑着:“嘛,我可不想有人站在我的头上。但又说回来,我可是在高一的时候就踩着一堆高年级坐上了副会的位置。要不是因为内海会长那时候在学生会和学生面前颇有威望,他现在也是我的部下。但是呢,因为我做事风格的问题,现在不少人对我有意见,所以为了确保下一届选举的万无一失,我需要和内海成为恋人来收割他那边的人脉。嘛,当然也不用真的当情侣,也可以像你们现在这样,当个假情侣哦~”

我猛地一惊:原来佐藤早就知道我们只是在假装了吗?!

佐藤看见我没有反驳,突然邪魅一笑,像是抓到了什么把柄。

“你看,内海会长找你做假情侣,不正是他拿你当练习品的证据吗?如果他真在乎你,他何不大方地和你交往呢?”

“他送了你一个发夹你就自以为是的沦陷了?这个发夹又值多少钱?就你会天真地以为他有多喜欢你了吧?”

“嗨呀,内海可真是罪孽深重呢!既把你当作自己的挡箭牌,又把你当做恋爱的练习对象…一举两得…最后你又得到了什么呢?”

“你只是个无人在乎的边缘人…只有学生会长在乎你…你不觉得很违和吗?你难道注意不到内海会长的内心在想些什么吗?”

“好好想想吧!雨宫ちゃん!不要天真了!赶紧从恋爱少女的天真想法中走出来吧!这个世界可没你想得这么简单!”

佐藤像只苍蝇一样,将学长的罪孽一遍遍地灌输到我的脑海里,动摇的种子悄悄在我的心里扎了根…

是啊…内海学长…为什么要找我…

佐藤朝通往食堂的大道望了望,似乎看见已经有学生吃完午餐往回走,于是便开口道:“唔,时间差不多了吧!你再不去找他说清楚的话,他或许就要吃完午餐走人了!好啦,快去吃饭吧~”

佐藤看着我仍不知所措地站在原地,满意地笑着离开了。

我望着佐藤消失在我的视野里,又将视线望向学长所在的食堂。是啊…学长还在等着我吧…

我迈开了步子,原本该二人世界般的食堂,现在却让我望而却步。

佐藤的话在我的心中萦绕着,如果继续和学长这样下去,学长一定会被佐藤陷害的吧…若只是陷害我,那我一定不会反对什么…可是学长要是因此受到牵连,那才是我最无法原谅的…呵呵…明明我除了学长外别无所求…这根本就不是在交易吧!这只是单方面的掠夺!但是如果学长真如佐藤所说,只是和我随便玩玩的话…那我又何必坚持下去呢…嗯…毕竟到头来也只是假扮情侣…那样的话…至少让我知道学长的心意吧…知道后我就会退场…

\newday{\rain\rainy}

雨开始一滴滴地下落,落在我愤懑不安却又无可奈何的心上。

\cutlineg\rain

“今天美月来的真慢呢…”内海看了看食堂正中央的大钟,又转头看了看窗外的天空,“唔…好像已经开始下雨了啊…要不要出去找一下美月?”

内海早早地买好了午餐,还没吃一口,只是在等着一个女孩的出现。等了十多分钟后,终于看到了熟悉的身影。

内海:“美月!”

内海朝我招招手,我在原来的位置上坐下。

“…”

“怎么啦,美月?发生了什么?你怎么全身都湿透了啊!?”内海从口袋掏出昨天一起买的手帕递给我,而我接过它后却只是将它紧紧攥在手心里,任由雨滴从发丝间随意地滑落。

我:“学长…那个传闻你应该知道了吧…”

内海突然安静了下来。

内海:“那个传闻…嗯…我知道…但是,这不是我们约好的嘛?”

我:“约好什么…?”

内海:“就是那个啊…假情侣…?”

我突然阴沉下脸:“为什么要是假情侣…”

“诶?”内海不清楚发生了什么。

我站起来将学长的手帕摔在餐桌上喊道:“为什么只把我当假情侣!”

内海被惊呆了,我的声音招惹来了食堂里所剩无几的学生的目光。

内海想要解释,但又不知从何开口:“因为…那个…”

我用着异常冰冷的声音说着:“因为你从没有真心喜欢过我对吧…?”

“这…不是…美月你先冷静点…这毕竟还是吃饭的地方…”内海慌忙地站起身,将手伸过来想拉我坐下,可被我一把拍开。

承认吧学长,让我能够死心离开吧。

“学长的回答只能是「是」或「不是」。”

内海沉默了。

沉默了呢,学长…

那就是承认了吧,学长…

我为自己这些天的真心感到可笑,心灰意冷地沉下眼去:“连佐藤学姐都知道我们是在装情侣了…看来是内海学长早就告诉过别人了吧…告诉别人我只是个冒牌的女朋友…随时都会被踹开…!”

内海惊恐道:“不可能…这件事没有任何人知道!”

我:“呵…那佐藤学姐又是怎么知道的?你觉得我还会信你吗…?”

已经将话说的这么绝,应该够了吧…接下来就是我的离场宣言了。

我尝试着平复心情,掷地有声地:“那好…内海学长…那我就告诉你……”

\highlight{“我——!从来都——!没有——!喜欢过你——!”}

心中有什么裂开了。说出来就好了,结束吧,这份还没开始的初恋。

我转身便要走,内海赶紧大步过来抓着我的手:“美月!”

我没有回头:“内海学长,以后请叫我雨宫。”

我用力甩开内海的手。果然啊…学长现在连抓紧我都做不到了…真可笑。

当我如释重负般地冲出了食堂,谷雨此时已如倾盆般下着,将我的制服连同内心一并清洗。雨声阵阵,将撕心的哭声连同那美好的回忆一并掩埋。眼眶模糊不清,却无法再分辨那是泪水还是划过脸颊的雨滴。

\cutlinea

\newday{\hate\rain\rainy}

看着雨宫离开后,佐藤从角落里走了过来,仿佛刚看完一场好戏。

我捂着额头愁眉苦脸:“是你做的好事吧…”

佐藤奸笑道:“看得出来你们是真心互相喜欢呢~说实话我当初真以为她是你用来对付我的假情侣呢~”

我没心思看佐藤一眼,内心只有一个疑问:“你是怎么知道我们是假情侣的…”

佐藤坐到我的对面,十指交叉搭着下巴说道:“雨宫ちゃん的内心想法都会写在脸上呢!明明一开始还在袒护你~可是一提到你们是假情侣,她就仿佛失了魂一样~于是我就趁机添油加醋了一把,她不一会就崩溃了!”

佐藤随后拿起我餐具里的苹果咬了一口后说道:“要不是雨宫ちゃん的出现,我还真难抓到你的把柄呢!所以啊,我建议你从现在起也老实点,毕竟搞她可比搞你容易多了~”

我虽然熟悉佐藤的手段,却从没想到她竟会做到这一地步:“你到底要我怎么做…”

佐藤:“从一开始不就跟你说过了吗?跟我做情侣,直到下一届选举结束。”

确实,这就是佐藤一开始的目的。她只是想把学生会长变成自己的一枚棋子。

我只好用右手揉着太阳穴:“行…我可以答应你…但我有一个条件…”

佐藤:“你先说说看~?”

我最终用凶狠的眼神瞪着佐藤:“不准再对雨宫さん出手…”

佐藤诡计得逞地笑了起来:“没问题!作为回礼,在场的所有人都会对刚才发生的事情守口如瓶,我的诚意如何?”

我苦笑着:“真是万分感谢!”

雨声轻诉着别离。如初春般突然降临的恋情,也在春夏交替之际悄然落幕。我承诺与佐藤成为「情侣」,将雨宫的一切抛至脑后。谷雨过后,夏日将至,我们终究不再见。

\cutlinea




%\character{佐藤 鳳(さとう おおとり)}{
%	\高光{出生日期}&  &未知\\
%	\高光{参加的社团}&&学生会\\
%	\高光{}&&\delete{弓道部}\\
%	\高光{喜欢的东西}&&未知\\
%	\高光{讨厌的东西}&&未知\\
%	&&\delete{难道要写我吗?}\\
%	\高光{性格}&&强势、咄咄逼人\\
%}
%{
%	你问为什么有这么多未知?对不起,我可没偷看过她的学生档案!
%	
%	那我为什么还要写?
%	
%	就给你看下模板啦!之后的介绍卡不会这么简陋的!
%}


\newpage
%\jelly——未完待续——\jelly





