\subsection{前夜祭}
\newday{\sunny\sunset}


{\filcenter{\highlight{\large{“学————长————!}”}}

}

下课后,我喘着气急匆匆地跑向学校门口,瞧见学长一下扑进学长怀里,尽情地享受着肩膀上只属于他的味道。

内海红着脸:“不用跑这么快吧!再说…还有其他学生呢…”

不知在他怀里蹭了多久,他才拍拍我的背,把我从身上拽下来。

“抱歉~老师班会聊了点文化祭的事情。”

清水高校(\japan{しみずこうこう},也就是我们所在的高校)的班会是在每周一的最后一节课之后,所以每周一的放学时间都比其他日子晚一些。而距今天起两周后的周六,就是正式的文化祭了,于是班上商量着文化祭要准备的活动。这将是我高中生活的第一次文化祭,却也是学长的最后一次文化祭。

内海挠挠头:“没关系…毕竟今天有班会嘛。”

我:“再说被别人看到又没关系!反正学长和佐藤さん当时肯定也是这样卿卿我我的吧?!”

“我…我哪有和佐藤さん卿卿我我?”内海慌张地辩解,看我仍然一副半信半疑的表情,只能继续辩解着,“真没有啊…!美月是知道的吧?!我和佐藤さん就不是那种关系…!”

我当然知道学长和佐藤的真实关系,不然当初或许就不会答应和他交往。只是对佐藤的怨气,总得找个地方发泄,而我的初恋男友正好被我选中了而已。

我挤弄着眉头:“所以在学长心里,美月还是初恋女友吗?”

内海用手摸着后脖颈,支支吾吾着点点头:“嗯,肯定美月才是我名正言顺的初恋女友啊…!”

“那就好!”听到这话,我才打算放过学长,奖励了他一个微笑。

“哟!小情侣真恩爱哇!”内海身后突然路过一位男生,用力地拍了拍内海的后背。

我一眼就认出来了这位男生,兴奋地喊道:“樱木学长!好久不见!”

内海用手捂着后背,咬牙切齿道:“朔,你啊!很痛诶!”

樱木无视了内海的怨气,朝我比了个YES:“雨宫ちゃん,好久不见!”

内海:“不要无视我啊!”

“呐!雨宫ちゃん,你们文本祭打算举办什么活动?!”樱木依旧无视了内海,顺便还跟我们一路走了起来。

“诶?”说起来樱木学长是有些自来熟,不过也有可能是学长经常在他那谈起我?无法确定是哪一边,樱木就这样自然地开启了话题,“他们商量的结果好像是办一个印章收集活动,找几个人穿指定的服饰散落在校园里,然后让顾客去满学校的跑。他们说这样就可以一边逛文化祭一边帮班级干活了。”

“诶?!听起来还挺有趣的诶!”樱木抬起热情的视线,重新聚焦到我旁边那位一直被无视的男生上,“内海,我们这三年好像还没见过哪个班级搞这个吧!”

“唔…”内海在脑海中努力地搜索着,“好像确实没有。”

樱木:“我就说吧!我们到时候比谁收集地快吧!”

内海:“这有什么好比的…?真是奇怪的胜负欲。”

“可是这可是雨宫ちゃん班级的活动诶!你不想表现一下吗?”樱木示意地朝我皱了皱眉。

内海这才回过头来,发现我正瞪着大眼睛看他:“哎,行吧行吧,至少我会去玩一下的。但是我可不想跟你比谁快,我还没那么闲!”

樱木用左手锤着右手手心:“哦!对吼!内海还有其他事情要做来着!”

虽然我知道学长要去帮佐藤帮忙,但这应该都会在文化祭之前就完成才对,樱木学长指的应该不是这件事。

“诶?所以学长的班级是准备办什么活动?”

还没等内海开口,樱木已经抢过了话茬:“我们班啊!一开始要说办什么电影主题,也就是放一块投屏然后一直放选定的电影直到文化祭结束!直接被半票否决了,超搞笑的!”

“真不知道他们怎么想出来的这么偷懒的活动!仔细想想的话,这完全只需要一个工作人员就够了吧!”樱木正说着,内海插进话题,“现在这个桌游主题还算是能让同学有点参与度。”

樱木:“内海,你快点把你的会长思维扔掉吧!文化祭不就是来偷懒玩闹的吗!尤其是我们都高三了,工作人员越少不就越好吗!大家都能腾出更多的自由时间!你现在可是有女朋友的人,不应该大力赞扬这种偷懒行为吗!”

内海被怼得哑口无言,身旁还传来女孩银铃般的笑声。

内海:“也不能是这种偷懒法吧!搞得像美月她们班一样轻松一点不就好了?”

我:“对呀对呀!我们班的超轻松的!”

学长趁机揉揉我的头发,惹得心里痒痒的。

内海:“但是我比较担心常见的桌游他们不一定想玩,不常见的桌游教起来又没耐心。”

樱木嬉笑着说道:“内海又在杞人忧天了!”

我:“但我觉得学长说的还挺有道理的…我平常就不太玩桌游,也不想听那么长的规则。”

樱木突然想到什么:“那!内海,你还记得3B班是什么活动吗?”

内海:“奶茶咖啡厅?”

樱木双掌一拍:“对啊对啊!他们班的客人要是光吃吃喝喝不是也很无聊吗?!为什么不直接把点了餐的顾客拉过来?这样他们又有吃喝又有得玩,吃着喝着耐心不就有了吗?”

“唔,朔,你这个想法还不错嘛?!”内海摩挲着下巴,“要不这样吧?我明天直接去找他们班,问问能不能把两个半的活动合并了,这样也省得他们埋怨我们抢走客人。”

樱木突然垂下个脸:“但是说起来,奶茶咖啡厅我们高一就办过了,没想到他们高三又来了一遍!而且都是没有女仆装的,这也太遗憾了吧!”

内海摇摇头:“那你还是别想了,他们班女生占比可是55$\%$,只要女生全体反对,根本不可能成功。”

“所以这不完蛋了嘛!?高中三年的文化祭没见到女仆装!这只能用灾难来形容了吧!这可是高中生涯的大污点啊!”樱木抓着头发,誓要把它们拔光的气势,随后灵光一现,朝我这边望来,“雨宫ちゃん班里的活动是穿着各种衣服在学校里游荡对吧!要不要考虑一下女仆装?!就当是我这位学长的一生的请求了!”

我满脸问号地看着樱木学长口里冒出来的、如此廉价的「一生的请求」,不过我倒是更在意某位学长的看法。

我将身体微微前倾,歪着头让发丝自然地垂下,呈现出一副人畜无害的俏皮模样:“学长想看吗?美月的女仆装?”

内海眼神游离着,唯独不敢看向我:“如果美月想穿的话…我也不反对就是了…?”

樱木捂着鼻子撤退了几步:“唔…恋爱的酸臭味…”

内海:“喂!这不是你先提出来的话题吗!”

我:“那学长就是想看咯?”

内海:“毕竟…是女朋友的话…这么想会很正常吧?!”

樱木:“哇…恋爱的酸臭味啊…!”

内海:“喂!朔!你是不是有点太吵了?!”

“对不起!打扰了两位的私人世界!”樱木用着完全不知错的语调捉弄着内海。没走几步,樱木朝左侧指了指说道,“内海!雨宫ちゃん!我要走这边了,再见!”

看起来樱木就是快要跟我们分开,所以才故意气一气内海。

内海这才消气似地跟他挥挥手。我被学长挡住,只好从他的背后探出头挥了挥手。

目送了一段樱木后,我和学长朝着另一个方向,也就是车站的方向走去。因为没有了多余的家伙,学长偷偷地牵起了我的手,而我也紧紧地回握着,将手指扣进他的指间里。唔,毕竟这已经是习以为常的事了,毕竟,我现在已经完全不会脸红了哦!…应该吧?!

最初他的步调总是比我快出许多,后来发现我总是没跟上,他便总是照顾着我缓悠悠地走着,就如刚才樱木学长在时一样。如今牵着手,他想走也走不快。我们就如此默契地走在回家的道路上。

学长想起时便会跟我聊起他高一高二时,这些店铺是什么样的。哪些店家换了人,哪些餐店口味变了,之前是什么装潢,现在是什么氛围。有时候他带着我绕一下远路,也会顺便带我去吃一些甜点,或者爸爸妈晚餐不回来吃的晚上,会带我去家庭餐厅吃饭。他对这条街的熟悉程度让我清楚地认识到他是我的学长这一事实,而我也为我们此刻能看见同一光景而庆幸。

这到底该怎么说呢?对了!就是和第一次约会的感觉一样!嗯…所以我们一起回家的时光,也能算是约会吧!?不过,我能比学长多欣赏一样东西,那就是学长在夕阳下的侧颜。这是学长看不到的,只有我能欣赏到的风景,这让我觉得自己一定比学长还要幸福那么一点点!——

“美月?你又在傻笑什么呢?”

“…啊!那个…!学长头上有一片落叶!”我拍拍学长的头,假装上面真有一片落叶——但不幸的是总是会被他抓住把柄。

站台上,一眼望去成片的藏青色,这时我才会稍微允许学长的手闲置一会——虽说我是在校门口扑到学长身上的那一方。但是容我解释一番的话,在校门口那样是因为半天都没见到学长了,难免会有些激动!但现在我们已经牵了一路了,学长能量已经补充完毕了,所以…!嗯,很合理。

内海从书包里拿出一根球形棒棒糖:“诺!老师给我们发的,好像是他朋友婚礼上顺回来的。”

“诶?!那学长自己吃啊!美月又不是小孩子了!”但是即使是大人也可以吃棒棒糖!这也是为什么我说着这些话,还是自然地接过了棒棒糖含了起来,是草莓味的,“说起来,樱木学长刚才说的,学长文化祭要忙什么呢?听你们聊起来,怎么感觉也没什么事呢?”

正好列车进站了,我们上了中间的一节车厢后,内海才接着说道:“唔,就是那些吧…你想要是真的要和隔壁班合作的话,我作为代表还是得去帮他们一些忙的…”

我:“啊!对了!学长班级高一的时候是奶茶咖啡厅吧?所以学长也会做奶茶吗?”

内海:“那个很简单的啦!就是记住配方然后混合在一起就行了。反正比起做饭来是简单不少的。”

我:“唔,那学长怎么也不想着来家里的时候给美月做一杯!”

“因为我没准备过原料,所以不知道怎么买…而且奶茶挺容易胖的…”一旁的水母开始嘟着嘴一脸怨气,显然这番话并没什么让她满意,内海只好摸摸后脑勺投降,“好啦!我到时候去问问吧。下次给你做…”

“这才对嘛!”

电车到站,走出检票口后,我们又自然地牵起手来。

“学长想吃青苹果雪糕吗?”

“上周不是才吃过吗?”

“有意见吗?”

学长没辙,只好牵着我往雪糕店走去。

“老板!要一根那个青苹果味的雪糕!”内海把我晾在门口的公椅上,自己进去买雪糕,出来后把雪糕递给我,“给。”

“只有一根吗?学长不吃吗?”我接过雪糕含进嘴里。

内海像那个夏天一样坐在旁边的公椅上:“你那个日子快来了吧,只能让你吃半根,剩下的归我。”

我停下摆动的双脚,一脸震惊地问道:“啊!学长怎么知道?!”

“猜的。”内海笑笑,一脸不想告诉我的模样。

“这已经是侵犯隐私了吧!会被逮捕的!”

“抱歉,请美月警察原谅我的过错!”

“那学长先说从哪知道的?”我想起前天发生的事,像是抓住了线索般质问,“不会是姐姐泄密的吧?!”

“没啦。”内海稍稍停顿,害羞着说道,“就是前天看到你买卫生巾了。”

“那也不一定就是这几天啊?”我据理力争。

“所以才说是猜的呀!毕竟细想一下的话,这三周美月好像确实还没来过。”内海双手合十一副忏悔的样子,“好啦,美月大人不要生气啦~”

“嘛,我也没生气来着。”想着之前也有过贪吃而痛不欲生的经历,犹豫了一下,还是把半根雪糕递向学长,“既然知道了就好好记住这个日子吧!”

内海凑过嘴来叼住我手中的雪糕,然后才抬起另一只手握住木签:“唔,美月咬过都不凉了。”

“那请吐出来。”

内海笑了笑继续啃着雪糕,不过因为已经快化了,所以几乎是狼吞虎咽似地吃完的。

内海看着手中的木签:“唔,这种雪糕好像没有再来一根的活动吧?从来没中过!”

我:“绝对没有!暑假吃的那两箱也一根都没中过!”

说着,内海站起身将木签扔进店门口的垃圾桶,回来牵起我的手继续往家的方向走。

内海:“那就送美月到这了?”

“嗯。”我们在牵手的终点站前面对面站好,“学长明天想吃什么?周六买了挺多菜的,要找个人分担一下。”

内海:“唔,都有什么呢呢?”

“土豆、茄子、胡萝卜、洋葱、鸡肉、猪肉、西红柿…”

“好了别念了…都行吧!就做美月想吃的吧!我跟着清盘就行了!”

我坏笑着:“那炒胡萝卜!?”

内海不敢相信自己的耳朵:“喂!那么多食材啊!就炒个胡萝卜吗!”

“开玩笑的啦!给你做青椒肉丝和章鱼香肠啦!”

学长也跟着我笑了起来。但其实我还有要对学长说的话,从一开始就一直在找机会说,就连刚才也是刚想提起又被其他话题打断了。此刻不知如何开口,气氛突然变得有些尴尬。

内海挑了挑眉:“好啦!那我差不多该回去了。”

他摸摸我的头发,准备转身离开。

“嗯…等下!”

我拉住他的手,他回过身来看着我。

“有一样东西还没还给学长。”

随着地上传来制服包落地的声响,我搭在了他的肩上。

再次睁开双眼时,我的鞋跟又踩回地上,低下头说道:“勇气…还给学长了。托学长的福,和姐姐关系变好了。”

内海缓过神来,温柔地捏捏我的脸颊:“美月是靠自己才和姐姐和好的吧?我什么都没做哦。”

“虽然姐姐也不告诉我学长到底说了什么,但是学长跟姐姐说了美月的好话吧?”

“都是一些简单的事实而已。”

“像是说美月霸道之类的?”我一脸怀疑地笑着。

“那也是事实吧?!”

“才不是事实!”我嘟着嘴撒着娇,心里却笑吟吟的。

我松开学长的的肩,捡起制服包用双手庄重地拎在裙子前,扬起一副满足的笑颜:“那学长再见!”

“美月明天见!”

\cutlinef{那是个青苹果味的吻}

\newday{\cloudy}

隔天早晨,第一节国语课后,早纪她们一如往常地凑到我的身边。

“真好啊…美月ちゃん最近和内海学长关系越来越好了呢~!”早纪先是摆出一副生无可恋地模样当着我的面跟静羽和初花抱怨着,紧接着又摆出一副遭了晴天霹雳的惨样,“昨天放学…我!又!又!又!撞见美月ちゃん和内海学长在校门口秀恩爱了!哇!杀了我吧!”

静羽拍拍早纪的肩,跟着摆出一副「没事,我懂」的复杂神态。

我摆摆手:“那个…我…下次注意…?”

初花掺和进来:“我好像没怎么遇到过诶?”

“谁让你每天都去篮球部练到那么晚才回家啊!”早纪叹叹气。

初花:“怎么说我?早纪你不也平常很晚才回去吗?”

早纪:“没有啊?轻音部一周就两三次团体排练,平时都是回家单独练习的。”

静羽:“哎,哪像我们写真部啊,没有任何强制措施,不想去就回家接我弟。我应该是撞见他俩最多的人了吧!”

早纪贴近抱住静羽一副同甘共苦、欲哭无泪的样子。

我:“诶?静羽さん还有弟弟吗?”

静羽叹了口气:“嗯,不过是继母的孩子。”

不知道早纪为什么突然插进话题:“我有个哥哥诶!说起来,美月ちゃん是不是有个姐姐来着?”

我瞪大眼睛:“啊?早纪さん怎么知道的?”

早纪一副说漏嘴了的样子:“嘛…那个…就碰巧知道的,毕竟你也知道,我的消息比较灵通。”

初花在背后戳穿道:“早纪的唯一消息源就是轻音部的那群家伙。”

早纪立刻回击:“怎么啦!初花的八卦源不也是篮球部?”

“不,等下…”我急忙打断,以免她们又科插打诨,“轻音部又有谁知道我有姐姐啊?”

早纪:“这个很复杂啦!”

最后,早纪就靠着很复杂之类的话,终究不肯说出信息源。我记得校园档案上好像也写着我有一个姐姐,但是应该不可能是佐藤泄露的吧?

初花看着早纪无奈地摇摇头:“话说回来,美月さん,内海学长他们准备文化祭办什么活动?”

早纪:“对啊!肯定比我们的活动强吧!”

我托着腮回想着昨天傍晚的记忆:“好像是玩桌游?外加奶茶咖啡厅?”

静羽:“诶?居然是两个活动吗?!”

我摇摇头:“不,学长说他们打算和隔壁班一起合作。他们本来是只玩桌游的。”

早纪:“呃啊…怎么感觉比我们班的还无聊…?”

我:“大概是早纪さん也不爱玩桌游的原因吧?”

静羽:“我倒还蛮有兴趣的!呐,我们到时候一起去玩吧?”

早纪:“到时候再看吧,轻音社那里可能还要彩排…歌曲太多了…”

想着到时候我会直接出现在他们班,我不知该点头还是摇头。

早纪突然微笑着朝我使了个眼色:“不过等我忙完我一定会去蹭奶茶的!”

我:“蹭奶茶…?!”

早纪:“诶?难道美月ちゃん没有特权吗?”

我:“但奶茶不是学长班级的活动啊…!”

静羽:“哎呀!让学长请客呗!我们可是去捧场呢!”

我:“啊…算啦!我会请你们喝的啦…!学长应该会挺忙的,就不要麻烦他啦…!”

三个人听完突然离我疏远了一些,毕竟有两个人还残留着心理创伤。

我:“哇!干嘛啦!”

早纪$\&$静羽$\&$初花:“干~嘛~啦~”

真让人火大!

“烦!”

一众人终于停下笑声后,我收起嫌弃脸问起早纪:“那个早纪さん,我们班级的活动会用到女仆装吗?”

早纪摆摆手:“完全不可能!这个我们肯定会全票否决的!对吧,静羽初花!”

初花:“那帮男生真是烦人!满脑子都是黄色废料!”

“诶?但我觉得还好啦,毕竟我们学校的制服不就这个长度?”静羽用手比划了一下自己的裙子,“女仆装的裙摆不是差不多到这吗?”

早纪:“哇,静羽!?你又来叛变了吗?不一样好吧!这就像是你在班级里穿泳装和去沙滩穿泳装的区别!大家都穿就等于大家都没穿!只有我们班穿,那肯定就是吸引来一堆色色的目光!”

“原来还有这种理论吗?原来如此,是吸引力的问题啊。”静羽又开始了思考,“那这样想的话,如果大家都没穿衣服…”

“我建议你立刻闭嘴。”早纪用手一把堵住静羽的嘴,然后转头用着一副令人发怵的笑脸问道,“所以美月ちゃん怎么想问女仆装?不会也要投赞成票吧?”

“没,不会…只是顺便问一下。我一定会投反对票的!”我疯狂摆着手,心里打消了投赞成票的念头。

午餐时学长告诉我,他已经和隔壁班的文化祭执行委员谈拢了,通过看电影的方式吸引旅客,顺便提高奶茶和咖啡的销量。他一边说,我一边把多出来的章鱼香肠和青椒肉丝夹到他的套餐饭盒里。

内海嚼着香肠:“真想不到啊,3B班一同意合作,第一件事情居然是重新投票决定要不要办女仆咖啡厅。”

我:“为什么?”

“因为我们班男生多,他们班女生多。昨天有说吧?他们班昨天女仆咖啡厅没通过就是因为女生基本都投了反对。”内海一边在饭盒里划着米饭,一边解释道,“但是现在的话,像我们班,就比如朔那家伙,做梦都想搞一次女仆咖啡厅,双手双脚都能投票的话,他绝对会四票赞成的。”

我把口中的香肠咽下去问道:“但是学长高一不是也没办成吗?不是说男生多吗?”

内海摇摇头:“因为那时候女生多,后来有几个女生调到B班去了。”

“那为什么今年不趁机再办一次女仆咖啡厅呢?”

“我们高二的时候,女仆咖啡厅就有70$\%$的投票率!但是被班主任一票否决了!说我们高一办过了,高三就更不可能了!”内海义愤填膺,但是很快冷静下来咳了咳,“哦,当然我倒是没什么所谓。”

“呃…不用特意强调最后一句。”我突然有点理解了初花的想法,“所以你们两班投票结果怎么样?”

“最终就是51$\%$的支持率通过了女仆咖啡厅。”内海像是看热闹不嫌事大般地点点头,“不过女生那边抗议说也要让男的穿女仆装,或者就是一半的女生穿执事服,一半的男生穿女仆装。总之是一些细枝末节的事情了。”

“所以就是41:80?”

内海点点头。

我突然开始在意一个问题:“所以…学长投了什么?”

内海下意识地变得慌张:“美月,这个其实还有中立票的,所以就算41票少了我,女仆咖啡厅还是会进行下去的。”

我直勾勾地盯着他的眼睛:“哦,所以学长就是41票中的一个吗?”

内海一语塞,无奈地捂着额头叹道:“美月怎么总是这种时候这么敏锐。”

“因为我听说男生脑子里都有些黄色废料。”

“咯…”内海皱紧眉头,像是在忏悔,“…美月有所不知,3B班男生因为比例少经常被女生们欺负,我这时候该帮他们一把就该帮一把吧?再说你也知道,朔他们对高一的事情耿耿于怀,两边的男生这次可是空前的团结啊!”

“好啦,不用解释了,我又没说学长什么。”学长慌张狡辩的表情并不常见,所以不禁觉得有趣。

“而且也说了,那帮女生肯定也不会善罢甘休的,最后肯定男生那边也会出什么幺蛾子吧?”

学长又开始喋喋不休起来,我随意地应声附和着,把吃不掉的菜都丢到他餐盒里。他倒也一边说,一边若无其事地夹着我递给他的菜。


\cutlinef\timepast

放学后,一道穿著白色室內鞋的身影穿梭在连接高一与高三的楼梯上,此时学长已经不在3A的教室里了。

“初次见面,我是1B班的雨宫美月,请问我能在文化祭的时候过来帮忙吗?”

“什么,要来帮忙?”

与我交谈的人是3B班的文化祭执行委员,他双手插兜倚着教室门,右脚交叉着脚尖点地,一副惬意的样子:“姑且提一嘴,我们计划是和隔壁班一起办女仆咖啡厅加陪玩桌游来着。雨宫さん应该知道吧?”

我点点头。

他有些不好意思地笑笑:“所以…女生是要穿女仆装的,知道吧?而且其实工作人员还挺多的,所以可能也不需要那么多帮手。”

“唔,是嘛。”

我耷拉着眼睛不知所措。

他叹道:“真没想到她们之前投的都是反对,但是一通过了反倒积极起来了,真是不可思议啊。说着什么「既然事已至此,就好好享受最后一次文化祭吧」之类的,真不知道她们在想些什么。”

此时他的身后冒出一个女生:“你要是能知道就见鬼了!”

“部长!?”

这个学姐我认识,她是现任料理部的部长,名字叫香织(\hl{かおり})。

女生把男生挤到身后,笑眯眯地点点头:“美月ちゃん,你怎么跑过来了?”

委员:“喂,现在是我在跟学妹聊天好吗?真没礼貌诶!”

“那我们先来聊聊你和内海くん私下勾结的事情怎么样?”部长握紧拳头冒着青筋,一副想杀之而后快地威胁道。

委员不自觉地想要后退一步:“等下,你不是也说要穿女仆装吗?”

部长:“谁让你们已经决定要穿女仆装了!想帮忙不就只能这样了吗?而且不是说好了男生也要穿执事服吗?你别反悔就行。”

委员:“我真不理解你们为什么会对执事服感兴趣,那个穿着很羞耻好吗?!”

部长指着他的鼻子咆哮道:“所以我们为什么要穿女仆装啊!你们男生就只想着躲在后面享受是吧?所以才说不要办女仆咖啡厅啊,都是你们这群男生不行啊!”

委员举起双手:“好好好,我投降,反正这事目前就这么定了。执事服和女仆装我都会去准备的。”

不知道怎的,他们就这样在我面前吵起架来。

部长啧了一声,换回笑脸看向我:“所以美月ちゃん也打算来帮忙吗?我倒觉得可以帮你插进去哦。”

我喜出望外:“真的吗?”

委员:“等等,这是学妹诶?不是我们班的来帮忙,有点怪吧?”

部长:“所以才说你不行啊,她是内海くん的女朋友,我这么说你能听懂吗?”

“啊?!”委员后知后觉,敲击着掌心惊呼,“哦,原来你是那个雨宫美月啊!那就没问题了!”

“我好像有自我介绍来着……”

%但是我对他们「为我是学长的女朋友」而改变态度这一点,稍微有些不悦。

部长生无可恋叹了口气,又转向我,模仿着委员靠在门沿上:“内海くん说他周日下午要去忙别的事,所以美月ちゃん,我把你和内海くん都排班到早上怎么样?下午美月ちゃん就去玩就好了,反正内海くん也不在这。”

我:“嗯?学长他有什么事?”

部长诧异道:“嗯?我还以为美月ちゃん会知道来着,他也没跟我说。”

我:“是学生会那里的事吗?”

委员堵到另一侧门沿摇摇头:“不可能,他看到佐藤さん就头疼。而且那家伙一副守口如瓶的样子,原来连自己的女朋友都没说啊。”

部长食指抵着下巴仰面猜测着:“但我想应该是跟美月ちゃん有关的吧?感觉他最近来找我都是来问你的事情。”

我腼腆地笑笑:“诶嘿嘿~”

委员:“我只知道他最近经常和朔在商量着什么。”

部长:“那就是轻音部的事情咯?”

委员摊摊手,表示不知道进一步的消息:“或许吧?他们也没告诉我,毕竟高一之后我们就不在一个班了。”

我:“唔,樱木学长也是轻音部的吗?”

部长:“「也」是指?”

我摆摆手:“我的意思是我也有个朋友在轻音部。”

“这样啊,是哦,樱木くん高一就一直在轻音部了,可能会认识你那个朋友哦。”部长突然留意到墙上的看了看手表,“不过时间也不早了,内海くん应该等挺久的了吧?”

我朝着部长的视线望向他们教室的大钟:“哇!那个,今天谢谢二位了,我要先走了,再见!”

匆匆鞠躬后,我刚跑出一步,又想起什么转过头来求道:“那个!今天我来过的事情不要告诉学长好吗,拜托了!”

他俩点点头,我笑着回应后赶紧跑回教室取书包,而身后的两人仍然闲聊着什么。

\argue{香织(かおり)\hfill 秋远(あきと)}{
\A 那孩子和内海くん的关系真好呢~

\B 我说,香织,不要在学妹面前那么损我吧?

\A 不满意吗?你先看看自己和内海くん做了什么吧。

\B 你不是也带头提议要男生穿执事服了吗?绝对是想看我穿吧?!

\A 你可太臭美了,只是看不惯你们的做法罢了。

\B 那你怎么不对内海くん发火啊?真不公平。

\A 哦,那我干脆去跟他谈恋爱好了?

\B 他已经有女朋友了。

\A 你再还嘴?

\B \tired…好了,不跟你吵了。

\B 说起来,1B班,朔的妹妹也在那吧?

\A 是嘛?我和樱木くん又不熟。

\B 唔,总感觉世界真小啊。

\A 你又在感叹什么。

好了,时间不早了,我们也回家吧,还要去补习班。

\B 哦,等我下!}

\cutlinef{“学长!”}

我如昨天一样扑到学长身上,他的肩上仍是我熟悉的味道。

过了一会,内海把我拽下来:“美月今天也好迟。”

“嗯…班主任又开了个小班会聊文化祭的事情。”我撒了谎。

“唔,那你们班主任可得小心点了,在课表外占用学生时间曾经是被家长投诉过的。”

内海聊起曾经某个班的任课老师因为拖堂被学校批评的往事。

我:“这…这么严重吗?”

内海:“当然。大家本来都是有各自的事的,要是因为突然的拖堂耽误了,那麻烦就不小了。”

看来下次不好用拖堂扯谎了。

内海牵着我的手回家:“所以呢?你们班今天聊了什么?”

我突然有些慌了神:“啊!就比如像是服装要准备什么之类的?”

内海:“哦…这种事直接放下周讨论不就好了?还专门过来占用时间。你们班主任真怪。”

我:“就是说啊!好怪啊!”

对不起班主任。

为此,我必须快速把话题混过去:“所以!那个!樱木学长今天没和学长在一块吗?”

“哦,那家伙今天要去社团。”内海回过头好奇地看着我,“怎么聊起他了?”

“因为!哦!说起来,最近学长和樱木学长是不是在偷偷地商量什么?”

“啊?我和朔?谁说的?”内海惊慌地用手指着自己问着,“不会是朔找你说了什么吧???”

我倒是好奇起学长的这幅神情:“没!就听料理部的部长八卦起来的!”

内海好像松了一口气:“那没什么吧?我和朔本来就一个班啊,有些事情要聊不也很正常?”

我突然发现用部长作借口是个不错的主意,既可以隐瞒我去找3B班执行委员的事,又不至于被学长怀疑。

我:“哦!说起来,部长还说学长文化祭下午有事要去忙?是要去干嘛?”

内海的表情泰然自若:“还能干嘛?难道美月不用我陪吗?”

我心里又欣喜又遗憾:“诶?就这吗?我还以为是什么呢!”

“不过因为被3B班报复,所以文化祭的上午不得不去帮忙。”内海突然想起什么不好的回忆,“说起来,今天下午的时候,3B班的执行委员跑过来跟我说已经决定了男生要穿执事服,真是恶趣味啊!”

“明明是男生先强迫女生要穿女仆装来着?”

“话是这么说啦!”内海低头叹着气,“早知道这样就不去找3B他们班了。”

我亲昵地贴到学长的臂膀上:“好啦!能者多劳嘛!到时候美月也会来看学长的!”

内海眼神充满着抗拒:“不要啦!等我忙完去找美月就行了!”

我如昨天一般撒起娇:“诶?难道不想让美月看嘛?明明昨天还说要看美月的女仆装来着?”

内海慌忙狡辩:“昨天是美月先问我想不想看的吧?我又没主动说想看?”

“又在转移话题!反正只要学长穿执事服,美月必须看到!”

“不要!那我到时候看到你来就躲起来!”

“哼!以为美月会找不到学长吗?部长可是站在美月这边的,问问她就知道了!”

“哦?真的吗?你的部长可还欠我人情呢!”

“哦!说起来还没算学长私底下找部长的账呢!”

“怎么说的像我在干什么坏事一样?”

我们一路吵吵闹闹一直拌嘴到上了电车。

与周一的晚霞不同,今天的夕阳还未染上茜色,天空仍是晴朗的一片。还未到高峰期的电车上,我们很轻松地找了一节无人的车厢。他望着窗外,我轻轻地靠在他的肩膀上。回过神时,已经没人在意刚才的吵闹是如何收场的。

我们家到学校途径五站,只消约十分钟的时间,靠在学长肩上更觉得是一睁眼的功夫。小憩之中,腹部突然传来一阵不适感。

他拍拍我的肩,示意我到站了。他牵着我的手拉我起来,身体却只觉得疲倦。

内海在前面领着我:“唔,时间过得真快啊,等到下周,学校里大概就会到处都是准备文化祭的匆忙身影了。美月到时候可得好好体验一下,真是青春的感觉啊。”

“嗯…”我没什么力气,只是更紧地握住他的手。

内海:“回想起去年在各个社团间跑来跑去,被其他班的干部吆喝来吆喝去的,真是够累的。今年总算可以稍微享受享受了。”

“呜…”

内海转过头来:“怎么了美月, 无精打采的…”

随着内海突然转身, 我也终于失去了力气,抓着他的手倏地蹲在地上。

他蹲下来焦急地问道:“美月?怎么了?身体不舒服?”

“呜…”我说不出话,只是一只手捂着肚子,另一只手紧紧抓着他的手。

“…还能走路吗?”

我铆足了劲摇摇头,肚子撕裂般地搅动着,眼泪不助地滚落下来。

他看了看四周,一片空旷,离家也还有五六分钟的路程,似乎也没有别的更好的方案了。

他把我手中的制服包夺去背到他的另一个肩上,将背呈现给我:“上来,我背你。”

我忍着剧痛从后方环抱住他。

学长抓着我的大腿站起身,小心地抬了抬姿势:“帮我看一下包,不要让它们滑下来。”

我不记得自己是否给了他回复,但他说完便迈出了步伐。他走得连贯而沉稳,并没有太多的颠簸感。他的背宽厚温暖,就像一床软绵绵的被子。后脖颈发出的味道也是如此令人安心。分散注意力后,仿佛缓解了一些疼痛。

脑子空空的不知过了多久,再睁眼时,学长已经把我背到了家门口。

他把我轻轻放到门口的台阶上,失去他的背后,我又蜷缩成一团。

“美月的钥匙我记得是放在最外层吧?”他放下我的包问道。

我无力地点点头,不过在我点头前他就已经找到了我家的钥匙。

开门后,他直接架着我的双腿把我抱了起来。明明是第一次被公主抱,可我却本能地知道要把手搭到他的脖子上。

他把鞋子随意地踩开丢在玄关,进门把我放到沙发上,把我左脚的鞋子摘下,顺手拿起几个抱枕给我抱着。

我的双手死死地抓紧枕头,像是要把它给撕碎。

他消失了一会,回来时跟我说着:“你有一只鞋子掉半路了,我去帮你捡,稍微等我一下。”

好想说「不要走」,但是却没有力气说出口。听着一串钥匙的叮铃声,学长飞快地跑到玄关,再是包被扔到玄关的声音,随后一声重重的关门声,最终家里安静地只剩下烧水壶的声音。

渐渐地,我除了肚子的疼痛外,失去了其他知觉,就连烧水壶的声音似乎也消失不见了。直到有人把我扶起来,我才缓缓睁开眼。

内海搂着我的肩,把一杯茶递给我:“来,把这个喝了。”

我缓缓伸出双手接过有些滚烫的茶,里面是殷红的一片,浅浅撮一口,红糖的甜味盖住了生姜的火辣,似乎还有一丝蜂蜜的回甘。囫囵地饮尽,肚子里暖暖的。不知是否是心理作用,疼痛似乎减轻了一些。

他把我捧着的杯子取走,摸摸我的头:“真乖~好点了吗?”

我点点头。



“那我再帮你接杯热水。”

他正起身要走,我急忙伸手拽住他的袖口,他见状又坐了下来,侧身看着我搞不清状况。

我丢掉抱枕紧紧地抱着他,把脸埋进他的胸口,他失去重心倒向沙发的另一边,我顺势把他压在身下,不准他离开。

“…美月?”

不想说话,就只想这样抱着他。鼻尖抵在他的衬衫上,隐约能闻到他掺杂着些许汗液的体味,此刻只想用他的气息覆盖掉疼痛感。

他见女孩不愿理他,只好把手搭在我的背上轻轻地抚摸着。

为什么我愿意把自己的软弱毫无保留地展现给他呢?或许是因为他从来都不会拒绝我的软弱吧。他总是回应着我的软弱,总是接受着我的任性。但是他从来不向我索求什么,所以只好我去向他索求,牵手也好,拥抱也好,接吻也好,明明我也希望他多主动一些,讨厌他总是这样保持距离,再多把我当女朋友一些啊!偷偷去找部长,在我不知道的地方和佐藤待在一起,为什么我比学长小两岁呢?想要和学长一个班级,想要在下课后偷偷看着学长,就算跟樱木学长说话也要跟我报备才行!好想再多一些相处的时间。所以只有这个可以尽情任性的时刻,绝对不会让你逃走的。

满脑子想着他,像是要把恼人的经痛从脑海里排挤出去。在内心不断的埋怨声中我昏睡了过去。

\cutlinef{\zzz}

\newday{\cloudy\sunset}

直到不适感侵入我的梦境,我才疲惫地醒过来。此时窗外已经染上了金色余晖。他胸口处的衬衫有些湿润,他就这样抱着我不敢轻举妄动地睡着了。

我从他的双手中挣脱开来,他也顺势被我惊醒,有气无力地问道:“哦…美月,醒了?”

我看着他点点头。

学长双手撑着沙发上起身,肩松垮垮的,一副刚睡醒的样子。

“肚子好点了吗?”

“嗯…谢谢学长。”我点点头答谢他。

下体稍感闷热,不好意思地压低声音问道:“学长…能帮忙拿一下包吗?”

“嗯,知道了。”

他笑着起身走到门口,把我的包拎到我面前,我从内部的夹层里拿出一叠日用卫生巾。

“那个…去下卫生间可以吗?”

“要我抱你过去吗?”

“…”

他浅浅笑着,无心的一问竟让我慌张地不知如何作答。正思考着要不要厚着点脸皮答应他,他倒已经脱下外套站起了身,把我横抱了起来。

“等等…还没答应呢!”

心跳瞬间加速,比起刚才更能意识到在他怀里那种安心感,夹杂着害怕掉下去而抱得更紧的依赖感,与不想过分依赖而产生的抗拒感。各种心情瞬间打翻在小小的心脏里,吵闹个不停。

他站稳身:“毕竟美月不拒绝就是想要吧?”

“才没说想要…”

我将他温暖的颈部搂得更紧些,脸埋进他的锁骨,不知道是不是真的害怕掉下去。

他平稳地走着,在他臂弯里,就像是漂浮在救生圈上。卫生间在客厅直往里走左拐的不远处,漂着漂着,救生圈便漂到了目的地。

他缓缓地放我下来,待我双脚触地,他才慢慢地将手从我的膝窝抽离。他扶我站稳,我才舍得松开他的脖子。

“稍等一下…很快就出来。”

我让他在门口等我,锁好门,蹲到马桶上一检查,满满当当的一片红,近乎溢出…

从厕所出来,见他正双手背在身后,倚在墙边一脸担忧的样子,见到我又摆出微笑走过来扶我。我来不及多想,上前一把抱住他,再次把他推到墙边。

他呆了一会,将一手抚在我的背中央,另一手温柔地顺着我的脑袋。埋在他的胸口看不到他的表情,但总觉得是一张宠爱的脸,看了只会让人更加心痒痒。只是紧贴着抱着,仿佛已抚平了腹部的疼痛。

“肚子还疼吗?”

“疼…还有一点点疼。”

“只是一点点吗?”学长紧紧把我抱在怀里继续安抚着,“美月一直都在揪我的肉呐。”

我这才意识到自己本能的行为,带着歉意地松开他的肉,在同样的位置用指腹轻轻揉了揉:“唔…对不起,疼吗?”

“肯定没有美月疼~”

我或许是下意识地把学长当成枕头了吧?过去也经常用蹂躏着枕头的方式来缓解疼痛。

学长顺着我的毛发继续说道:“但是美月想揪的话就揪吧,我想了解美月的感受。”

我埋在他的胸口,用力揪起一块肉回应他,然后又松开轻轻地揉了揉,能听到他轻微的笑声。

我回想着昨天他对我说的话,委屈地道歉:“对不起学长,昨天不该吃冰激凌的…”

“我也有错吧…但是下次就不准吃了~”

我捏着他的肉,既是赞同又是抗议。

内海:“美月平时会吃止痛药吗?”

“布洛芬嘛…也不是每个月都要吃,再说现在吃也来不及了。”

“是嘛~”

学长一直抚摸着我、安慰着我。我烦闷的心也逐渐放松了下来,就像瘫倒在一张暖呼呼的床上。

“说起来学长居然知道红糖姜茶?”

“嗯,之前妈妈也会在我生病的时候帮我泡,也不难。”

“但是红糖对那个作用不是很大…”

“但是甜一点会好喝些吧?”

“那倒是…”

我从他的怀里侧过头,察觉到天空似乎更加阴沉了些,相处的时间消磨殆尽,得赶紧让学长回家了。

“学长该走了…”

“美月已经没事了吗?”

我在他的怀里点点头:“而且爸妈快回来了…”

我失落地松开学长,躲到他身后不舍地把他往玄关推。

来到玄关,他拎起地上的制服包,背到肩口,看了看我,顺手捋了下头发,额间垂下一绺发丝。

“那美月注意休息…我走了。”

“路上小心。”

他微笑着与我挥手,转身似乎也能看到他同样失落的表情。

他关上门后,我转身倒了杯热水一饮而尽,回到客厅躺到沙发上,一只手揪着抱枕,另一只手伸到抱枕下揉着肚子,才刚消停的心脏又开始跳个不停。

\cutlinec

\newday{\cloud}

一周后的周二,初花终于有一天不用训练,早纪便又提出放学后去哪里庆祝我脱单。

我:“要不去家庭餐厅?”

我跟她们比划了某家家庭餐厅的位置。

初花:“诶?原来这里还有家庭餐厅吗?我以前居然都不知道。”

静羽摊摊手:“我估计是内海学长告诉美月さん的吧?”

“我估摸着也是。”早纪叹着气附和道,“那就这么定了吧?”

早纪总是我们中间那个做决定的人。

中午与学长打了声招呼,说放学不能一起回家了,学长倒是爽快地答应了,说自己正好还有些文化祭的事情要处理。就这样,我们四人便一起前往家庭餐厅。

早纪与初花在前面走着,我与静羽在后面跟着。

早纪的头发比我稍长,今天也如往常扎着单马尾,衬衫的最上方扣子敞开着,水色领结耷拉在领口外面,背着一把乐器;初花则是短发,总是敞着外套,没穿针织外套也没戴领结,一副刚运动完的模样;静羽整体与早纪有些相像,只不过是披肩散发,穿着倒和我一样得体,胸口挂着一款佳能相机。

回想起来我们很少有过这样一起回家的经历。

我盯着在我面前晃悠悠的乐器包问道:“早纪さん,文化祭的时候你负责哪部分乐器?”

早纪挠了挠颔:“我?唔,除了最后一首都是吉他。最后一首好像不用我上场。”

初花在一旁一副惬意的样子:“你哥终于舍得疼你了吗?”

早纪针锋相对道:“切,你在胡说什么?少弹一首就叫心疼了吗?”

我好像想起上周听说樱木学长也在轻音部的事情,顺口问道:“那早纪さん认不认识一个叫樱木朔的学长?我听说他也在轻音部。”

早纪惊恐地侧转身问道:“诶?美月ちゃん怎么认识他的?”

我:“因为是学长的朋友啊。”

初花走在前面不回头大声地喊着:“那就是她哥啦。樱木朔和樱木早纪,没发现他们名字都很像吗?”

我突然发现了什么:“好像是同一个姓诶?”

静羽在一旁附和着:“不止!朔(\japan{さく})和早纪(\japan{さき})都是樱木(\japan{さくらき})平假名的一部分,很默契吧?”

难怪第一次看到樱木学长的面容和听到他的名字的时候,有一种说不上的亲近感,原来是因为和早纪很像啊?!

早纪:“怎么你们都把该我说的话给说完了…”

早纪像是要强行转移话题似的开始埋怨起期中考试,大家不约而同地埋怨着数学考试太难,只有静羽不合群地说着数学考试出的太水,引来早纪的埋怨。初花又说到正好是因为期中考试大家才一致翘掉了社团活动,我们才有机会碰到一起。看来至少我们都喜欢考试后去放松放松。

\cutlinef{\timepast}

几个人漫无目的地走着聊着,不一会,我们便来到了约好的餐厅,服务员领着我们到靠窗的四人座,早纪将乐器放在座椅边上,和我一起坐到窗边,初花顺势坐在早纪的外侧,静羽坐在我的旁边。

这是一家夫妇开的餐厅,以甜点和常见的炒菜为主。室内柔和的黄色灯光配合着各种装饰画与顾客照片,与柔和的背景音乐一起,让人感到自然般的舒适。几盆养眼的植物和一架钢琴摆在显眼的位置上,很适合像我们这样的年轻学生。

学长告诉我,这里曾经火爆过一段时间,但终归由于位置偏远,如今又回到不温不火的状态。但我觉得这种氛围正好。

早纪拿起摆在桌上的菜单开始看着:“好了,那我们先点餐吧?”

初花也顺着拿起菜单:“哇,这的东西还挺丰富的嘛!”

静羽:“而且价格也不是很贵,想试试这里的布丁\pudding。”

初花:“那我点这个松饼吧,看起来挺有分量的\pancake。”

随后早纪点了份芭菲\parfait,我跟在后面点了一小份焦糖果仁盖层的冰激凌蛋糕\shortcake。

点完餐后,服务员收走菜单,早纪先摊着双手趴到了桌上:“啊!最近真累啊!”

初花也跟着瘫在一边表示赞同,我好奇地问道:“文化祭的事情吗?”

\saki:“当然啊!还有期中考!都是累人的麻烦事!”

我看着两人,思忖片刻,回想着之前的话题问道:“那樱木学长是早纪さん的哥哥的话,是不是知道内海学长最近在忙什么?”

早纪受宠若惊道地爬起来:“我…我不知道,老哥他没告诉我。”

\jelly:“所以他们真的在忙什么事吗?”

\saki:“唔…我感觉是有吧,最近老哥一副鬼鬼祟祟的样子,完全不知道他在忙什么。”

\ichika:“呃,所以才说早纪你是兄控啊,老是粘着你哥,都把你哥逼回叛逆期了。”

\saki:“叛逆期个鬼啊,我又不是他妈!”

早纪冲着初花喷沫,初花赶紧把身体倾到另一边,静羽看着她们无奈地摇着头。

\jelly:“唔,话说回来,所以上个学期早纪さん总是能知道内海学长的消息,就是因为樱木学长吗?”

早纪难为情地僵笑着:“毕竟他消息很灵通嘛!”

初花无心地插进话题:“说起来,我们班的活动只有美月さん和静羽被选去参加了吧?”

我叹了口气也瘫到桌子上:“好倒霉,都是人数不够最后变成抽签的错!”

静羽安慰道:“也还好啦,反正我们班的活动也很水,美月ちゃん是要穿小熊玩偶吧?”

“对啊…真不想让学长看到我那副模样,完全认不出是我!”说着我又叹了一口气,“静羽さん是海盗服吧?”

静羽点点头:“我觉得那个也挺傻的,还非要带个骷髅眼罩。”

过了一会,服务员依次端来餐点。

初花正拿起刀叉,静羽赶紧伸手拦在她面前:“慢着,不拍照吗?”

早纪:“对啊对啊,拍个照先!”

我把盘子递过去,初花一脸嫌弃地照做。

那两人一人抓着手机,一人抓着胸口的相机拍了起来,确认效果后终于点点头:“嗯,搞定~”

初花:“不用每次都这样吧?就只是甜点而已。”

早纪:“哎,初花啊,美月ちゃん没手机就算了,你怎么老是对美好的事物没有一点反应呢?我们之前就猜,你是不是打篮球打傻了。”

初花面朝早纪喊着:“静羽她是写真部的,文化祭要办展览就算了。我倒是很好奇早纪你一个天天玩摇滚的,怎么这时候又装起女孩子了。”

早纪轻轻地猛拍桌子:“什么叫装女孩子?我本来就是女孩子啊?!”

初花阴阳怪气地敷衍着:“好好好,你是,你是!行了吧?”

早纪:“呵。”

早纪的脸上画着淡妆,眉毛靓丽而不突兀,脸上总是带着令人舒心的笑容,看上去确实是会对拍照之类的事情很感兴趣的类型。总感觉像是没有染发、也没有耳钉耳环、也没有跟着潮流说一些晦涩难懂流行词汇的辣妹角色?相反,初花的脸色总是冷冰冰的,叹气、皱眉、嫌弃是她的脸上最常见的表情。

初花说话间的功夫,松饼已经一半下了肚,放下刀叉放缓进食速度。剩下三人倒是在慢慢品尝着餐点。

早纪尝了口芭菲,捧着脸满心幸福,我跟着挖一勺冰激凌蛋糕,是久违的甜腻与冰凉。

“还好最近天气还没有太冷啊,不然绝对不能吃冰的东西了。”早纪舀着芭菲里的冰激凌如此感叹道。

“是啊,还好今天内海学长不在,不然他绝对不会让我吃冰的。”我用着舒适的频率点着头,对早纪的话深表赞同。

早纪:“哇,内海学长怎么管这么多?这是剥夺人享受快乐的权利啊?!”

我尴尬地笑笑:“因为之前喝常温牛奶拉肚子,还有吃冰激凌痛经什么的,好像都被他发现了。”

“呃,痛经啊,那确实挺受罪的。”

静羽吃着布丁赞同着。

早纪跟着点点头:“确实啊,看来都是美月ちゃん自己的问题啊,那也没办法了。”

我焦急地大喊:“不要这么快就抛弃我啊!”

早纪笑吟吟地摆摆手,示意我不要当真。

静羽也吃了一半布丁的样子,放下勺子拍拍手说道:“所以我们该进入正题了吧?美月さん,你和内海学长应该也交往差不多一个月了吧?有什么感想?”

早纪也跟着放下勺子搭话:“对啊对啊!美月ちゃん,快来点恋爱八卦!”

我不知从何说起:“该怎么说呢?就挺好的感觉?”

“呃…”

早纪露出了一刹语塞的神情。

\shizuha:“那我换个问题吧,内海学长有没有惹你生气的地方?”

\jelly:“唔…应该没有吧?”

我想自己的一些小心思应该还不能算作是生气吧?

\ichika:“真稀奇,如果是篮球部那些男生,我每天都会生气不止一次。”

\saki:“内海学长毕竟当过学生会长,人品应该没得说吧?”

早纪举着勺子分析着。

\shizuha:“但是也还是很难得吧?一个月居然没有惹你生气的时候?”

我摆着手:“真的没有啦!而且我和学长每天待在一起的时间本来就不多嘛。”

早纪伸出手指点了点:“午餐的时间,放学的时间…唔…还有什么?”

“还有每周六中午?”

“哦,周六中午…”早纪才反应过来什么,“诶?休息日你们还在幽会吗?”

静羽一手托腮用着妩媚的眼神看着我:“你们都去哪玩了?”

我:“没有出去玩啦…!就是他会来我家。”

初花:“这可不得了,居然把男生叫到自己家来!爸妈不会说什么吗?”

我慌忙摇着手:“爸妈那时候不在家啦!”

早纪不知在思考什么,随后摆出看似平易近人的微笑问道:“呐,美月ちゃん!说起来,内海学长有没有腹肌啊?!”

“诶?!我想想…”当初帮他换衣服擦身子的时候好像看到过,当时也偷偷摸了一下来着,“我记得有一点吧,但不是很凸显。”

早纪突然捂着嘴大笑起来,眼睛眯成一条缝。

我不解:“怎…怎么啦!学长有腹肌有什么问题吗?”

初花也是一脸迷惑:“对啊,有腹肌对很多女生都是加分项吧?连我也有。”

静羽闭上眼优雅地推了推眼镜:“美月さん,她应该是在笑你看过内海学长的身体了吧。”

我想了想静羽的意思,反应过来后心跳到了嗓眼,噌地拍桌而起:“啊,真是的!早纪さん!我们还什么都没做过啦!”

早纪笑得一手捂紧肚子,一手软绵绵地指着我:“我不行了!美月ちゃん着急的样子好搞笑!\laughtear”

我憋红了脸看向初花,同样指向早纪:“初花さん,能帮我个忙吗?”

初花心领神会:“乐意之至,我也早就在想这么做了。”

初花把早纪压到身下,开始泄愤般挠她的痒痒。

早纪:“哇!哈哈哈!不要挠那里!要死了!哈哈哈!快住手,初花!”

静羽赶紧起身拍了一张早纪凌乱的照片:“这张不错,放展览那应该会很受欢迎吧?”

静羽拍完坐下,看着相机如此说道。

我跟着坐下凑到她身边,她把相机凑到我俩中间,我看着照片忍不住笑出声。

早纪赶忙铆足劲推开初花,爬起来大喊着:“这是公开处刑啊!我头发都被这家伙搞乱了!”

静羽:“开玩笑的啦~最多就发给你哥看看。”

早纪:“那更不行!绝对不行!快把相机给我,我要把它删了!”

早纪憋红了脸要过来抢相机,静羽赶紧高呼着哄道:“逗逗你的呀~我哪次真发给过你哥!”

早纪似乎也同意了静羽的说法,没好气地坐到位置上:“行,记住别让我在其他地方看到这张照片。”

静羽像是刚经历了一场浩劫,赶紧带着早纪转移话题,开始认真地询问起我和学长发生的各种事情,像是上周六他在我家一起做料理部新学的章鱼烧,或许最近他带我去过的餐饮店,当然还有痛经那天的事情,只不过自动省略了那些过于暧昧的部分。

早纪:“哇!内海学长还背痛经的美月ちゃん回家啊!我要是被我哥当众背回家,想死的心可能都有了!”

静羽:“早纪你说这话你自己信吗?”

初花模仿着小女孩的语调嘲讽道:“「哥哥大人!你能背我一下吗?你问为什么…?因为…{\large{因为…!}}」”

早纪涨红了脸,把初花压在了身下,抓着她的肩膀不停地摇着:“我在死前要先杀了你!”

初花:“「哥哥大人!哥哥大人!救我啊!」”

静羽摇着头,然后看向我:“抱歉啊,美月さん,这两个人私下里就是这么闹腾。”

我憨憨地笑道:“挺有趣的,就像是好朋友一样。”

早纪从初花的身上爬起来:“那比起初花这家伙,我倒是更想和美月ちゃん这样的乖乖女做朋友。”

初花也爬起半身,手肘杵着双人椅探出头来:“是啊,这点我赞同。”

\newday{\cloud\sunset}
静羽无奈地看着两人,随后又看向我:“说回来,好像不久前佐藤会长还在抓校内情侣啊?你们听说这事了吗?”

初花朝着我说道:“哇,不会是准备蓄意报复你俩吧?”

早纪摆摆手:“不不不,我听说的版本是那对男女并不是情侣,而是男方在单方面强迫女方,所以佐藤会长出手制止了。”

我:“唔…佐藤さん真的会这么做吗?”

毕竟在半年前她是强迫我做某些事的那方。

静羽:“唔…是嘛?那我到时候再去打听打听。”

不知不觉间大家已经清空了餐盘。

或许是留意到甜品杯中溢出的晚霞,早纪双手一拍,露出灿烂的笑容:“总的来说,再次恭喜美月ちゃん脱单成功!和内海学长相处这么融洽也算是可喜可贺!那么吃饱喝足聊爽,我们也走吧?”

静羽和初花也道了声祝贺,虽说我觉得这事也没到需要大张旗鼓的地步,但还是跟大家道了声谢。

大家走到门口,初花最先与我们分开,我们剩下三人回家路上仍聊着文化祭的话题,随后早纪和静羽也走向另一条路。

此时形单影只的我,抬着头留意到茜色的夕阳,终于明白如果有相机的话,我一定也会像早纪和静羽一样拍下这张照片吧?

但并不是因为它的美丽是多么不同寻常,而是因为此刻我由衷地想将所见所感分享给某个人。

心里犯着痴,宣泄般地开始奔跑着,将朦胧的喜悦撒在这寻以为常的回家路上。

\cutlinef{\turned}

“会长!3B班和2A班都撞了女仆咖啡厅。”

“知道了…我明天去找他们商量一下。”

“会长!1F班的预算不够了。”

“让他们去找会计自己想办法解决,解决不了再来找我。”

“会长!学校的招牌…”

佐藤:“那个不是已经定好了吗?副会长没去通知你吗?”

“会长!Switter上已经超过1w转发了!”

佐藤:“别跟我汇报这些没用的消息。”

学生会干部走一个来一个,办公室的佐藤摊在椅子上,翘着二郎腿将椅前腿抬起,终于有了片刻安宁。我看着眼前的场景甚是欣慰。

“哟,佐藤会长正忙啊。”

佐藤仅用余光瞥了我一眼:“哦,杂鱼前会长啊,你来干嘛?今天没和她一起回家吗?”

“呃…我现在的称呼已经变成这样了吗?”我刚上前就被佐藤的称呼来了个下马威,“她找朋友玩把我丢下了。我就顺便来看看你的工作。”

佐藤:“有什么可看的?都是些鸡毛蒜皮的小事吧?”

我走到她的靠椅背后,靠在墙上问道:“舞台事宜,听说你有个大计划。”

佐藤:“是轻音部部长告诉你的吗?”

我应了一声:“你怎么会想到那种计划的?我还以为你也会像我们一样全校选举的。”

佐藤像是回味着说道:“还记得去年文化祭的时候,你建议两个高一的班级合搞活动的事吗?大概是那时候的事情给我的灵感吧?说起来,今年你也把3A和3B的活动合在一起了吧?”

我:“那件事你还记得啊?就是当时脑袋一热想到的方案而已。”

佐藤:“不是挺好的吗?除了最后颁最优班级奖出了一些争议。”

我叹道:“所以才说是脑袋一热想出来的,那样会与校方的一些流程起冲突。”

佐藤:“无所谓吧,重要的不是开心吗?你当时是这么说的。”

我哼了一声,事到如今我仍是这么认为的。如果只是为了一些奖项,彼此的眼里只剩下了竞争,那便丧失了文化祭的意义。我不喜欢争夺所谓的第一,我希望每个人都是第一。佐藤仍然记得当初我对她说过的这些话。

佐藤歪了歪脑袋:“反正我本来就是恶役上台,所以无论我做什么他们也不敢反对吧?”

“大家是相信你才让你当会长的,恶役什么的只是你的自嘲吧?”

“谁知道呢?”佐藤背对着发出似是而非的笑,“说回来,你邀请四ノ宮学姐了吗?”

我无奈地点点头:“嗯,但是真不想再跟她扯上关系。”

佐藤:“谢谢了,你不去邀请就得我去邀请了。”

我:“你怎么还不情愿的样子。你和四ノ宮学姐不是不熟吗?”

佐藤:“什么话?不就是因为不熟才不情愿?”

“那我倒更希望跟她不熟。”回想着曾经与四ノ宮发生过的一切,真是让人不堪回首。

一阵秋风袭来,我不禁打了个喷嚏。

“哟,那孩子在想你了?”

能明显听出她的笑声。

我吸了吸鼻子:“佐藤会长也喜欢开无聊的玩笑吗…”

“所以你真的要为那孩子做到这地步吗?”佐藤拿起舞台演出的计划表,翻看起上面的节目。

我:“只是表达一声感谢而已,有你说的那么夸张吗?”

佐藤:“让人有点火大罢了。”

我:“我替你被揍了一顿还不够你消火吗?”

佐藤:“这是两回事。况且本来就是你自己提出的这个计划。”

我:“至少让你成功当上会长了吧?烟草部的那些家伙现在也都在乖乖听你的话吧?”

“嗯,但是撑不了太久的。”佐藤低下头像是陷入某种回忆,“接下来就等明年那个人毕业了,是时候结束这件事了。”

我:“烟草部的事情,到时候就拜托你了,佐藤さん。”

佐藤用着不易察觉的声音回应着,片刻的寒暄结束后,佐藤下了逐客令:“好了,有棘手事我会去找你的,你就不要再来我这瞎溜达了。”

我叹叹气:“好好好,我可真是不受待见。”

我走到门口准备离开,又一位学生会干部跑了进来。

佐藤从干部的身后探出头来:“最后的文化祭…和她一起好好享受吧,我会让它变得盛况空前的!”

“嗯,拜托你了,现任会长。”

我挥挥手捎上了大门,门内又瞬间回到了文化祭准备工作的紧张氛围之中。

\cutlinec

\newday{\cloudy}

次日早晨的课间,3B班的文化祭执行委员拿着一本小册子把我叫出教室。

秋远:“雨宫さん,我马上要去订制服了,方便告诉我你的尺码吗?”

我:“唔…上衣S码,鞋子37码(23.5cm)?”

他手上唰唰地记着,随后从笔记本的中间取出两张照片给我,照片上面的是两款女仆装。

委员:“唔,关于女仆装的款式,我们决定的是闭襟、短袖、裙子及膝、粉蓝配色,袜子是白色过膝袜,就像照片上这样。这没什么问题吧?”

我看了看照片,胸口没有暴露,与之前私下看到的一些女仆装照片不太一样。

见我没异议,委员接着问道:“鞋子有浅蓝色或者浅粉色,雨宫さん选哪个?”

我不假思索地回答道:“浅蓝色!”

他又唰唰地动着笔:“顺便问问,雨宫さん有穿中跟的习惯吗?”

我想了想,似乎没什么经历,制服鞋是只有3cm的低跟,但我还是如此问道:“学校的制服鞋算吗?”

秋远:“哦,那就当没有了,到时候就给你准备和制服鞋一样的低跟吧。”

我好奇地问道:“其他学姐呢?都是中跟(5.5cm)吗?”

秋远:“嗯,她们都是现充嘛,还有平时穿高跟的。”

我:“那我也想试试中跟…?”

秋远权衡了一下:“唔…要不这样吧,我姑且给你备一双低跟的。明天放学后你来我们班试穿一下低跟或其他学姐的中跟,到时候看看哪个更舒服。如果以前没有穿女仆装的经历的话,也正好可以熟悉一下穿法。”

我:“好!谢谢!”

秋远:“唔,然后香织她们给你讲过接待客人的礼仪了吗?”

我点点头。

秋远:“那3A班那边有来找你教一些简单的桌游玩法吗?”

我摇摇头。

他叹了一声:“哎,他们可能是把你忘了。不过也没啥事,到时候雨宫ちゃん就接待客人和递餐点就好,这样也可以省点事。陪玩桌游什么的让学长学姐他们去搞就好了。”

委员写罢,朝走廊另一个方向指了指:“顺带一提,我们班还有3A班的活动被换到一楼了,也就是原来1D和1E班的位置。离你的教室挺近的,到时候别找错了。”

我点点头,委员便与我告了别,我朝他挥挥手,兴高采烈地回到教室\laughcute,心里越发觉得与学长的首次文化祭正在接近。

早纪趴在我的课桌前打趣道:“哇,美月ちゃん又一副痴女笑。”

我赶紧收起表情:“谁痴女了?!”

静羽:“美月さん,刚才那人是谁?”

我:“你说3B班的文化祭执行委员?”

早纪:“那个人我好像也见过,跟我哥关系好像还挺好的。”

初花:“我记得他们班是女仆咖啡厅吧?传得沸沸扬扬的。”

我:“是这样的,听说是近三年来第一次女仆咖啡厅?”

“等等…我好像猜到什么了…”静羽又开始如往常般推测起来,“早纪,你还记得美月さん当初提问我们班有没有女仆装的事情吗?”

早纪回忆了一下:“好像确实有这么一回事?所以呢…?”

静羽挑了挑眉看向我这边,我尬笑着错开她的视线,掩盖自己的心虚。

早纪:“不会…吧?!”

初花好像理解了什么:“哇,内海学长真差劲啊!”

但好像理解的不是很恰当。

我忙着解释:“不是那回事啦!”

\cutlinef\food

中午学长也如昨日般抱怨着3B班的执事服。

内海:“哎,看来逃不过去了,真要穿那种衣服了啊!”

\newday{\food\cloudy}

我:“是3B班的执行委员\hl{也}去找学长问尺寸了吗?”

内海点点头,他突然意识到什么不对劲:“美月怎么知道?”

我慌了神:“哦!那个樱木学长跟早纪さん说的,然后早纪さん告诉美月的!”

“什么鬼,朔那家伙跟他妹连这个都说吗?真是个妹控。”内海叹着气埋怨起来,“诶?美月已经知道他们是兄妹了吗?”

我夹了一块鸡蛋卷到嘴里:“嗯,昨天从初花さん那知道的,还说早纪さん是兄控。”顺便也夹了一块给学长,“话说学长应该早就知道了吧?怎么从来不告诉美月?”

“因为我还以为早纪ちゃん早就告诉你了啊?后来一问朔才知道早纪ちゃん一直没说。说到底,人家妹妹都瞒着不开口,我一个外人也不方便说什么吧?”

看起来早纪的话题成功让学长忽视了我刚才那句话的漏洞。

“倒也是,不过初花さん和静羽さん看起来都知道的样子,就美月不知道,有些难过吧。”

“那现在她们不是告诉你了吗?说明她们是真的把美月当朋友了吧?”

学长如是安慰道。

我撇着嘴想了想,似乎也没错。

“说起来美月,周日的文化祭会邀请谁来?”

“唔…昨天给姐姐打了个电话,她应该会来吧?但是爸爸妈妈那边就不好意思说了…”

“不邀请爸妈吗?”

“因为不想让爸爸知道啊…我和学长的事情。所以干脆妈妈也不叫来了,不然爸爸要起疑心的…总而言之他们也很忙啦!明年再弥补他们好了。”

我也是很犹豫的啊…!但是爸爸来了的话,就不能一直粘着学长了…明年学长就毕业了,到时候多陪陪他们就好了吧?

“说起来,学长那边呢?都邀请了谁?”

“唔…上一届学生会长和我妈吧?”

“上一届学生会长?”

怎么感觉好像在哪里听说过这号人物。

“嗯,毕竟这是传统嘛。刚下任的学生会长去请上一届毕业的学生会长。”

“听说是女生?”

“美月怎么连这都知道?”

“好像听谁说起过。所以明年就是佐藤さん来邀请学长吗?”

“嗯,道理上就是这样。”

“那学长去跟她说一声,明年文化祭我会来邀请的。”

内海笑出了声。

可是这样的话,怕是明年也不能邀请爸妈了吧…?不过也不一定能瞒到明年…

“那学长只邀请妈妈吗?爸爸呢?”

“不知道跑哪去了,应该来不了吧?”内海的脸上蒙上一层忧郁。

看起来学长的爸爸也是工作狂的类型?

我颇为好奇地问道:“学长的妈妈知道我们的关系吗?”

内海扬起嘴角:“早就知道了啊…她还很好奇为什么我们那么迟才确立关系。”

“还不是学长和佐藤さん的错?!”

从那天跟伯母的通话中就能感觉出来,伯母应该是知道我们的事情的,这也让我对这段感情多了些自信。

谈天之间,学长清空了最后一口饭,紧接着把手边的保温杯递给我,里面装满了牛奶。

回想起来,他开始为我准备热牛奶,大概就是自那次拉肚子之后吧?而且能明显地尝出来,里面的牛奶被偷偷换成了鲜牛奶。保温杯是双层真空的,捧在手心并不暖手,喝入口却格外暖和。

我咕噜吞下半瓶,啵了啵嘴,双脚不自觉地晃悠起来。

学长笑吟吟地看着我,指了指自己的上唇。我抿了抿,娇羞地拿出情侣手帕擦了擦,把剩下半瓶还给学长。

我:“给,一起喝吧?”

内海本想拒绝,却被我硬塞到手里。他看看保温杯口留下的唇印,又偷偷瞟我一眼,将唇贴在印旁边,交叠着喝掉了剩下半瓶。

喝罢,他才张大嘴吐着舌头:“好烫…!”

“不烫啊?”

“烫啊…!”

也是,我觉得不烫的东西在学长嘴里是烫的。但是他的体温却比我要更烫一些,真是不可思议。

%看着学长不占理的模样,我伸出手捏捏他的脸庞。
%
%内海没有闪躲,任由我捏着:“干嘛?”
%
%我尽情地捏了个爽,收回手指碾了碾,回味起昨天傍晚的场景:“学长的脸还挺嫩的,跟小孩一样。”
%
%内海不明所以:“什么?这是在夸我吗…?”
%
%“算是吧?”
%
%我痴笑着,学长也跟着宠溺地笑起来。

吃完饭我们又一起去食堂后方腻歪了一圈。

随着日子的堆叠,这些也渐渐成为熟悉日常的一部分。而非日常的是,学校门口的校匾被换了写有「清水祭」的款式,似乎也能看到佐藤一行人出没的身影。台阶的踢面装饰着鲜丽的贴纸,走廊地板上是为外来人员准备的引导标志,纸质流彩悬挂在天花板、教室的玻璃窗边,花形、蝴蝶形、心形的贴纸涂鸦更是随处可见。走廊上到处都是奔波的藏蓝色校服,他们的口中也无时无刻不在聊着文化祭。

在日常与非日常的交织之中,我们的第一次文化祭也悄然而至。

%\newday{\sunny}

\cutlinec

\newpage


\newday{\sakura}


\character{樱木 早纪(さくらき さき)}{
	\高光{出生日期}&  &\highlight{2018年8月4日}\\
	\高光{身高}&&\highlight{165 cm}\\
	\高光{体重}&&\highlight{51 kg}\\
	\高光{参加的社团}&&轻音部\\\hline\hline
	\高光{喜欢的颜色}&&\colorfy{redpink}{\color{white}{粉红色}}\\
	\高光{喜欢的食物}&&芭菲\\
	\高光{喜欢的运动}&&游泳\\
	\高光{喜欢的事物}&&音乐相关\\
	&&\delete{兄长}\\
	\高光{讨厌的事物}&&被其他人说是兄控\\\hline\hline
	\高光{日常装饰}&\高光{发型}&过肩低单马尾\\
	&\高光{装饰}&粉红的樱花发绳\\
}
{
	
}

\newpage

