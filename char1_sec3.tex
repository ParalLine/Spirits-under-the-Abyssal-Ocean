\newpage
\newday{\sunny}
\subsection{双月}

{\fillf{}\highlight{“诶{\Large{诶}}{\Huge{诶}}!{\Large{!}}{\Huge{!}}”}\fillf{}}

班级里几位女生如往常一样传来浮夸的震惊声。不过真说起来,也有些与往常不一样的地方…

早纪:“美月ちゃん和内海学长在一起了吗?!”

静羽:“唔,虽然昨天就觉得是迟早的事。但是不是太快了点?!”

初花:“确实。而且内海学长昨天都没来学校啊,雨宫さん难道和内海学长在放学后还见面了吗?”

静羽:“你别说,还真是!”

是的,不同的就是她们正在八卦的人是我。

我:“因为我离学长家还蛮近的…所以路上就碰到了…然后就不小心…?”

总之先随便糊弄过去吧,总不能告诉她们昨天抱着学长睡觉的事情吧…保不准会发生什么事情。

“但是真好啊!雨宫ちゃん总算是修成正果了吧?毕竟离水族馆事件也过去好久了。”早纪一边自顾自地感叹,随后又一脸好奇地问道:“怎么样怎么样!和内海学长谈恋爱的感觉!”

早纪一脸热情地盯着我,我连忙举手摆好架势怕她扑上来:“唔…很安心?很舒服?就像个小孩一样可以扑在他的怀里撒娇的感觉?”

“呃…”静羽此时一脸嫌弃地拉过旁边的初花,“美月ちゃん的表情好恶心…”

初花在一旁用能让我听到的声音窃窃私语:“恋爱的酸臭味…从她脸上溢出来了。”

我娇嗔着:“喂!静羽さん,初花さん,我听得到啊!”

静羽:“不然呢?不让本人听到还有什么吐槽的必要吗?”

我:“咯…”

当我把视线从那俩人身上移回来,却捕捉到了早纪脸上羡慕又惆怅、转瞬即逝的表情。当我一眨眼,早纪的表情又变回了温暖的笑脸:“那我们哪天一起回家替美月ちゃん庆祝一下吧?!”

静羽:“干嘛还哪天啊,要不就直接今天呗?”

早纪从我的桌前站起身,朝向另外两个家伙:“诶?不行,今天部长说要安排一首新歌来着,不知道是什么来头,明明文化祭的歌单很早之前就定好了的。”

初花也把静羽推到一边:“抱歉静羽,今天篮球部要训练来着。”

我:“抱歉静羽さん,今天要跟学长一起回来来着。”

静羽一瞬间感觉到了背叛:“哦对!我今天也要去漫研部来着!看来大家都没空呢!哈哈!”

早纪无奈地摇摇头:“所以就说「哪天」啊…或者就是等大家都社团活动结束之后。”

静羽:“所以说,主要还是初花的问题吧,篮球部天天训练。等你哪天不去了跟我们说一声呗?”

初花点点头。

早纪:“唔,那要是初花哪天有空,我也跟社团请假一天吧!”

我:“那我…?”

静羽:“美月さん,你不会想整天粘着学长吧?你的朋友会很伤心的哦!”

虽然我和早纪是更早些就开始打招呼了,但是与静羽和初花是昨天才开始熟悉起来啊?原来静羽已经把我当朋友了啊…

我:“好吧好吧…到时候我也跟学长说一声。”

早纪:“说起来,美月ちゃん没有社团吗?”

初花:“难道是归家部的?”

我:“唔,其实最近有在考虑加料理部…想给学长做好吃的。”

静羽伸出手掌阻止我:“美月さん,理由啥的不必特意告诉我们,特别是理由还和你的男朋友有关的时候。”

“喂,静羽,不要再打击美月ちゃん的恋爱积极性啦!”早纪一把拍掉静羽伸在半空的手,“对不起,美月ちゃん,静羽这个人不太看氛围。理由什么的都无所谓,开心就好。而且料理部我听说不是很忙,也不会太占用你们的私人时间。”

我点点头,笑着答谢早纪。

\cutlinef{“学长!”}

\newday{\food\cloudy}



自从我们确定关系后过去了一周,大概就到了十月初。今天的午餐时候是我先过来占座。学长看到我正摆着手,便拿着餐盒坐过来。

这些天学长的伤渐渐地恢复完毕了,昨天帮他确认了伤势后才允许他摘掉创口贴,现在基本上已经看不出伤痕了。

“哎,测验好累…”内海边叹着气边吃着午餐。

“毕竟学长已经高三了嘛~”我说着便偷偷将胡萝卜丝夹给学长。

渐渐地,学长也习惯了这个自己曾经最讨厌的食物。

我:“学长,那之后要不要找个其他位置一起吃午餐?”

内海托着下巴开始思考:“我觉得,哪都一样吧…?反正也找不到没人的位置…”

我:“唔,可是学长从教室过来很远吧~”

内海有气无力地搭了句:“还好…就当是散步了吧…”

我:“嗯,也是。回去的路上还可以绕点远路!那就还是在食堂吧~”

内海点点头。

能看出学长并没有太多干劲,感觉就像是被每天的考卷给榨干了。

我:“好啦,那多吃点吧~看你愁眉苦脸的~”

内海这才注意到自己丧着一副脸。

内海:“抱歉美月,感觉最近有点亏待你了…”

我摇摇头:“没关系,理解~”

每天能见到学长,心里就已经很满足了。我这么想着。

内海嚼着胡萝卜丝,想起什么似的说道:“说起来美月不是参加料理部了吗?最近学到什么菜了吗?”

我:“唔…好像上次去的时候,她们在准备做曲奇来着?好像文化祭之前都会一直教一些文化祭上会准备的食物。”

“所以美月知道吗?料理部每年文化祭都会和食堂合作摆摊,在中午和晚上摆摊,会准备各种烤肉啊,小蛋糕啊,章鱼烧啊,或者就是平时里食堂会准备的午餐…”内海回过神来,眼前的女孩已经眼里放着光垂涎三尺了。

我:“真的吗?!好吃吗?!好期待!”

内海宠溺地把我按下来:“当然好吃咯,毕竟食堂那边的叔叔阿姨也会帮忙,到时候一起去吃吧?”

我:“诶!说起来,早纪さん好像也说了轻音社要排练歌曲诶,文化祭还有音乐会之类的吗?”

“有是有,每年体育馆那边会被借来表演节目。就说说去年的情况吧。因为我们学校是两天的文化祭,周六下午2:00~4:00演出,以预热节目为主,一般是写真部拍摄的短片开场,然后依次是合唱部、演剧部或者其他班级组织的小节目;周日下午则是1:00~4:30,那天是以正式表演为主,同样也是写真部开场,之后依次是吹奏部、舞蹈部、演剧部的大型话剧和最后的轻音部,每个部门都给了40分钟左右,中间会有十分钟的小活动充当中场休息。等所有表演都结束后就到篝火晚会了,校长致辞、给最活跃的班级颁奖,最后亲自点燃篝火,在那之后每个班级找人收拾教室,其他人在操场上听着音乐跳舞,欢送夕阳下山,优哉游哉地文化祭就结束了。”

内海又啃了几口胡萝卜,“但是麻烦的是他们每年都会争周日最后一个时间段,毕竟这是文化祭最火热的时间段。去年吹奏部就和轻音部在那争,学生会就被拉着去调停。去年的前学生会长就教我直接去全校投票就行了,免得被斥责袒护某一边。结果就是轻音部获胜了。”

我托着腮帮子看着学长科普本校的文化祭:“唔,这样啊。硬要说的话,确实吹奏部的庄重氛围不太搭啊…话说学长去年要在学生会忙文化祭的事吗?”

内海一脸疲惫:“是啊…就像组织表演节目,然后还要分配场地,就像料理部那些,还有桌游部也会过来借场地。还会被找去帮各个班级的忙,帮忙装饰学校、装饰体育馆。只要一闲下来,就会有新的事情找上门来。”

我:“哇…好恐怖…所以学长今年可以轻松一些了吧?”

内海一脸惆怅:“怎么说呢…虽然确实主力已经变成别人了,但是还是要去作指导工作帮帮忙。毕竟新任学生会长没什么经验嘛。”

“那…能不能不要去…?”

看出我的表情有些微妙,内海放下筷子,双手交叉认真地看着我:“美月,佐藤さん毕竟也是学生会长,作为前学生会长,我需要去指导她,不然文化祭就有可能出差错,到时候如果大家不能尽兴的话,这就得不偿失了对吧。”

“话是这么说…”我有些委屈,用筷子随意滑动着米饭。

内海叹了口气:“好啦,我会尽快忙完那边的事情,然后就来找美月好吗?”

“说话算话!”

“说话算话。”

吃完饭,内海从口袋里拿出一个包装精美的球状物体递给我。

我一脸疑惑:“这是…?”

内海微微一笑:“给美月的信物。”

我打开一看,惊喜道:“诶,\colorfy{lime}{\color{white}{青苹果}}?!学长从哪里搞来的?”

内海:“从樱木家附近搞来的。”

我将青苹果捧在手心:“樱木学长嘛?…但是吃掉的话就不是信物了啊?!”

内海:“怎么会呢?吃下去变成身体的一部分,这不也是一种信物的存在方式吗?”

“嘿诶~也有点道理!”我咬了一口青苹果,酸甜可口的汁水就在心中蔓延开来,“吼契!”

“嘴巴里有东西就别说话了,别咬到舌头。”

我把青苹果递到学长面前,咽下嘴里的苹果后说道:“学长也吃吧,毕竟是信物嘛!”

“这不太好吧…美月不是已经咬过了吗?”

“什么意思嘛!亲都亲过了,还在意这个嘛!”

“不啊,但是…我的意思是,就是…那个啦…”

“什么啦!听不懂!快拿着,手举着很酸诶!难道是想美月喂给学长嘛?”

内海有些不好意思地接过苹果,犹豫了片刻,终于挨着青苹果的齿印旁咬下了一口。

“嗯…!吼甜,吼酸。”内海将青苹果递回来,我顺势接过来。

“不行哦学长,不要在嘴里有食物的时候说话哦!”  

内海有些吃了哑巴亏地盯着我,我则是用着故作高傲的眼神盯着他。在一阵沉默的对峙中,内海咽下了口中的苹果片,两个人终于忍不住笑出声来。

在一人一口地吃完青苹果后,我拉着内海在无人的小路上散步。这条路环绕在食堂之后,再加上我与学长本身来食堂比较迟,所以这时鲜有学生出没。这个时节秋风正盛,清凉之余夹杂着一丝寒意。

\newday{\maple\food\cloudy}


我和学长已经换上了一身藏青的秋季制服\uniform{blueunif}{whitel}
\dressdisplay{white}{bluesky}{blueriver}{yellow},除了领口和袖口,裙边和他的裤腿上都带着白二本襟线,整体显得没有那么死板。他的西式外套敞开着露出里面的白衬衫,如往常一样没有扣上最上面的扣子,也没有带领带;而我因为怕冷,外套的三粒扣全部扣紧,衬衫最上方扣紧的扣子外是女生统一的水白黄格纹领结,与衬衫之间还套了配套的针织开衫\cardigan{whitel}{bluel},萌袖和宽领口像小兔子一样从制服内探出头来。

学校规定女生的制服裙下面禁止套裤子,对袜子的长度也有很严格的要求,比如冬季最长的袜子只能是黑色的过膝袜。听早纪她们说,要是被男老师抓到,会被恶趣味般地强制要求当众脱下来,非常恶劣,似乎以前还有女生带头抗议过。现在还没到能够穿长筒袜的季节,所以我只能踩着白色纯棉小腿袜,任凭风撩过大腿间,我只能挽着学长的臂膀直哆嗦。

秋风清扫着落叶,却将落叶洒落地更加肆意。红色的枫叶、金色的桦叶满布在石砖路上、散落在空中、或仍顽强地挂在梢头,在暖洋洋的秋旭照射下,更添一层秋意。不过总归是到了秋末,没走一会我就被秋风冷得完全地黏在了比阳光更暖和的学长身上。

内海:“这么冷的话,就把裙子放下来呀!你看看你都卷了两圈了!”

我:“诶?可是这样好看嘛!其他女生都是这样的哦!”

这在一些爱美的女生中非常流行:将制服裙在腰部绕着腰处的皮筋卷两层,更方便地露出半截大腿。校规里写的是裙边必须及膝,将整个大腿遮住,但遭不住学校大半的女生都在违反校规,所以除了正式场合根本没有人管。我是不久前才跟着早纪她们学的,但是确实更冷了!

内海摇摇头:“不过说起来,如果实在觉得冬天太冷的话,是可以向相关老师申请换成男生校裤的。”

我:“诶?还有这种事情?我在别的高中也没听说过诶?!”

内海:“嘛,毕竟是去年才开始征求意见的,校规手册里还没正式录入,所以作为高一新生的美月应该是不知道的吧!”

我:“那学长想看美月穿校裙还是校裤?!”

内海看着我,脸上写满了纠结:“美月不是很怕冷吗?那要不穿裤子?我去帮你交申请函?”

“诶?学长不想看美月穿裙子吗?真扫兴!”我撒娇着把脸埋进内海的臂膀,拒绝与他对视。

内海:“诶?怎么怪我啊?可是美月就是很怕冷吧?”

我:“哼!那学长就多抱一会嘛!而且初中其实就已经冻习惯了…这里至少还能穿过膝袜!”

“行行行,那美月到时候别感冒就好!”内海只好无奈道。

“那就是学长答应了会多抱一会咯~”我轻轻地推了推学长,学长顺势往一边踉了几步,我被学长带着快要倒向同一边时,学长又将我推回来。我们就这样推搡着,像两个醉鬼似地随意踩踏着落叶,发出一阵沙沙声。打闹了一会,两人便又恬静地挤在一块。

内海:“美月还是一如既往地元气呐!”

“那还用说~!”我把学长的手抬得高高的,逗得他噗嗤笑了出来。

内海收了收笑脸:“谢谢,美月。”

我:“嗯?谢什么~?”

内海:“因为看着美月的样子,我稍微也恢复了些干劲。”

“或许是学长的青苹果在起作用哦~”

我挽着内海的手臂哼起歌来,希望能让他稍微放松下心情~

\cutlinef{“学长!”}

\newday{\cookie\food\sunny}



周六因为爸妈仍要上班,所以在我的邀请下,学长就会来我家和我一起吃中饭。虽说现在睡裙也收起来换成了睡衣睡裤,但平日休息在家起床第一件事倒是换身休闲的居家服。款式是妈妈挑的珊瑚绒质感的藏青居家服,蕾丝边的娃娃领显得整体没有那么死板,所以即使给他看也还算体面。因为平时都会清理地板,所以平常起床都会穿好厚袜子,更方便在家里乱跑也不至于弄脏脚和地板。 

“材料都买来了吗~?”

内海拎起手中的购物袋:“当然,午餐,还有美月要的做曲奇的原材料。”

我拍拍手蹦了蹦:“好呀~!好呀~!今天午餐打算做什么?”

我接过内海手中的购物袋低着头往里探,学长则趁机偷偷理了理我头顶还没睡醒的呆毛说道:“煲鸡汤。”

我正要拿回厨房,又想起什么转过头来问道:“今天应该不是我做菜吧?!”

我之所以这么问,是因为后来学长告诉我,他只会做家政课上教过的菜品,其他的他从来没去主动学过。结果就是好几次他来我家,都得我做菜然后他在一旁学。

内海噗嗤一笑:“今天我做!鸡汤我在家也会给妈妈煲,也不难。”

或许是因为知道我今天还想做曲奇,所以他就想着不能午餐和点心都让我一个人做吧?

我把他的食材放到岛台上,走到客厅拿着小水壶打算先浇会花。

学长拖鞋进屋后,将夹克挂在门口的衣架上,跟着我来到客厅。

“之前第一次(周六)来的时候,美月好像说自己一醒来就会浇花。”

学长跟在我身后如此问道。

“唔,是这样。怎么啦?”

妈妈教我一周浇两次,我一般都是周六和周三浇。

“但是好像每次我来的时候美月都在浇花?”

“因为学长来之前都一直在睡懒觉嘛~”

这些都是妈妈养在家的植物盆栽,它们散落在茶几、玄关、窗台、厕所还有二楼的卧室与阳台等。尤其是二楼的窗台,整齐地堆放着两排颜色各异的多肉盆栽。我正在浇的是摆放在电视机两边的两盆有着特别意义的花。

“嗷,正在给「美月」浇花呢~”

学长指着电视机右边的那盆蓝石莲说道。他第一次周六来我家时问起花的名字,我便跟他介绍了一些。

“嗯哼~好看吧?”

“嗯,比姐姐的那盆可爱。”

在电视机左边的是紫珍珠,那是属于姐姐的多肉。

这两盆是我们出生的时候,妈妈为我们挑选的。听妈妈说,当初买来的时候也是只有这么一小盆,如今老桩长得太茂盛已经被放到了二楼阳台,我和姐姐的房间以及这里放着的都是剪下来的分株。妈妈以前教了我们养护的方法之后,就让我们自己打理自己的植株,姐姐上了高中之后因为学业问题也是我帮忙照顾她的那株。等妈妈也变忙了之后,家里的植株基本都是我一个人打理。好在妈妈买的植株都不需要频繁照顾。

浇完一层的植株,又丢下他去二楼浇了一圈后,我才回到一楼。这时他已经在厨房那清理鸡汤的食材,我则跑到他的身旁用打蛋机搅黄油和蛋黄。

他往锅里加着姜片大葱段焯水,完了跑到我身边来围观:“美月要帮忙吗?”

“哇!不准偷看,这是要给学长的惊喜!”我赶紧用身子挡住学长的视线,一边往里面加低筋面粉和杏仁粉。

内海摊摊手,转身去处理要加的辅料。鸡汤要加的东西不多,泡好的松茸去根切块,胡萝卜切段,玉米切大段,最后简单冲洗下枸杞。处理好辅料后,也差不多焯完水了,将鸡转入大锅,加水没过食材,接着便开始炖煮。炖煮期间无事可做,他便又一次跑到我的身边来,此时的我才把原料混合到一半。

察觉到那边没了动静,我就用眼角瞟了一眼,然后把身后某位男友的视线挡住:“学长不准偷看哦!”

内海:“好好好,我不偷看!”

当我终于完成原料的处理,才注意到旁边已经摆好了裱花袋和裱花嘴。看起来是刚才,学长去储物柜里翻出来的。

“诶?学长知道怎么做曲奇吗?”我一脸震惊地看着学长。

内海点点头:“料理部的部长姑且也欠我些人情,所以就顺便问了问她怎么教你们的。”

我一脸失落:“那不就没有惊喜了吗?”

内海:“没事啦,美月想给我一个惊喜就已经是一个惊喜了。”

“在说什么?绕口令吗?”

“在感谢美月哦!”

“切,搞得美月像笨蛋一样。”我一边将面糊装进裱花袋,一边埋怨着学长,可内心却有一丝愉悦。

内海:“嘛,只是想知道最近女朋友在忙些什么,不行吗?”

我:“哼,就不能让女朋友有一些隐私嘛?”

内海坏笑着:“唔,不能说不能吗?(\japan{だめじゃだめ?})”

“不能!”

好像在不经意之间,学长又说了个绕口令。

“好吧,那下次我就不去料理部那边打听了,原谅我好吗,美月?”内海双手合十,向明明没在生气,却摆出一张臭脸的女朋友道歉。

“哼~!”

我没有理会学长,专心地将面糊挤出曲奇的形状,共计十五个曲奇,随后将它们放进烤箱里。

我盯着烤箱说道:“那个,美月没在生气哦。所以学长也不用道歉。只是…下次要是偷偷来打听的话,也请装得不知道好吗?美月想让学长看到自己努力的一面。”

“努力的一面吗?…”内海盯着盯着烤箱的我,随后又看向烤箱,“虽然知道美月想说什么,但是美月努力的一面,不是一直都能看到吗?”

“所以才说请装作不知道好吗…美月也想让学长能对自己心动一下。”

内海柔和地说道:“我现在就在心动着哦~美月想让我心动这一点,就已经让我心动了。”

我边听着,边将烤盘换了个朝向:“哼,油嘴滑舌的。”

内海:“哪有,明明很真情实感好吗?”

我:“因为美月现在还在生气。”

内海:“诶?不是说没在生气吗?”

“要我把所有事情都教给学长吗?有没有生气不会自己判断吗?”

“啊…现在绝对是在生气!”

“没有在生气!”

“果然在生气吧?!你的自称都变了!”

“怎么了?\highlight{我}怎么了?要我一直自称\highlight{美月}吗?!我就\highlight{我}了!\highlight{我我我我}…!!”

但是我绝对没生气,就只是有一点点生气…但绝对不是真的生气。嗯,这对学长来说或许还太复杂了,因为我自己也搞不明白。真麻烦…

“好好好,我投降,我要去煮饭了,就老样子?”

“嗯,老样子。”

老样子指的就是煮两人半份的量,我只吃半人份,学长吃一人半份,剩下半分就是应急用,一般是我们俩平分掉。学长绝对是找了个借口开溜了,算了,饶过他吧。

烤制曲奇的前期工作完成后,我也只需要等待就行了。这时我才把注意力集中到旁边的鸡汤上,此时学长正在往里加松茸、胡萝卜和玉米。

“好香啊~”

“还不错吧~想先尝下汤吗?”

“现在还没调味吧?”

内海点点头,用勺子舀了一勺递到我嘴边,另一只手挡在勺子的正下方:“嗯,就只有鸡汤味,不过应该还有点汤。”

我轻轻吹了吹,用嘴唇点了点勺中的鸡汤:“还好,不是很烫。”我慢慢品着鸡汤,学长则慢慢抬起勺子,将鸡汤灌进我的嘴里,“嗯,很鲜!”

内海用拇指摸了摸我的嘴角和下嘴唇残留的汤渣,然后送到自己嘴里尝了尝,看着这一幕,我的心情稍微有些悸动。

“学长也想尝尝鸡汤吗?”

“诶?不用啦,等调味的时候再…”

我懒得听学长说完,一下子堵住了他的嘴,然后又松开。

“这下尝到了吗?”

内海反应过来,就像是看着做了坏事的猫咪,立刻摆出一副宠溺的表情:“那我能再尝一口吗?”

“本来就是学长做的鸡汤…”

于是他用同样的方式,尝了第二口鸡汤。

我:“能尝出味道吗?”

内海:“可能得再好好尝一下…”

这次学长用手环在我的后腰,引导着我靠到他的身上。我双手不自觉地贴在他的胸膛,他突然的举动惊得我有些不知所措。我佯装着想要将他推开,却也没有阻止他将舌头伸进来…可是应该早就尝不出鸡汤味了才对!

“不…不准再说没尝到了…!心脏要受不了了…!”在学长的围剿中,我胆怯地提出异议。

内海挤着眉质疑:“不是美月先开始的吗?”

我:“但是本来就只想让学长尝一口的…”

内海嗤地笑出来:“那现在不生气了吧?”

我扁着嘴:“一直都没生气呀?!”

内海微笑着点点头,松开了在腰间的右手却没有抽离,这次我们应该是达成了共识吧!

我:“等下…曲奇,要夸哦…第一次做没什么把握。”

内海左手伸过来捏我的脸颊:“为什么要以「我不会夸」为前提呢?美月做的一定会很好吃的。”

我:“那是因为学长是男朋友吧,滤镜太重了。”

内海笑笑,随即烤箱的定时到了,我从学长的怀中挣开,将曲奇取出来,至少形状看起来没太大问题。

我:“最后就是等它们自然冷却,学长那边怎么样了?”

内海打开炖锅,往里加了一些盐调味,最后花了点时间调了碗鸡丝料汁。

内海:“快好了,米饭应该也快了吧?”

“学长多久前煮的?”

“大概就半小时前吧?”

“那就快好了。”

我从橱柜里取出餐具,学长也过来帮忙把一些果蔬盛好先端到餐桌上。期间米饭也好了,我盛米饭,学长则回来用一个大碗盛鸡汤和辅料。来到餐桌后,学长先让我坐下,顺带着帮我盛了一碗鸡汤。

“稍等一下,我再处理一下整鸡。美月先吃吧。”

我点点头,顺势夹了块玉米。

不一会,学长将撕好的鸡丝装在另外一个盘子里端过来。

内海:“怎么样,味道还行吗?”

“好喝!超好喝!”

内海看了看桌底下晃悠的双腿,看得出来女孩是真心觉得好喝。

我:“学长平时煲完鸡汤都会撕鸡丝吗?”

内海点点头坐到我对面的位子上:“不然吃起来很麻烦吧?而且煮透了很容易撕,不觉得很解压吗?”

我:“那学长辛苦啦!给学长夹点鸡丝!”

内海将自己的碗递给我,任我往里面乱夹食物。

“美月要酱汁吗?”内海接着从我这接回盛得满满当当的一碗鸡汤。

“不要,蘸着鸡汤吃就好了。”

简单餐前礼后(\japan{いただきます}),我们共进了午餐。

“说起来,美月上周去料理部,一直待了一个多小时哦。”内海尝着鸡丝,随口聊起一个话题。

“唔,好像是。怎么了?”

“我想之后美月去料理部的时候,我就在教室里等你吧,正好可以看会书。”

“哦,行呀~学长上次好像说是去轻音部有事情来着?”我顺口咬着玉米。

“嗯,那次正好有事,所以之后待在教室里也方便找到彼此。或者就是等差不多了,我来料理部找你。”

“不要!不准学长来偷窥!学长就待在教室吧,美月会带回些点心作补偿的!”

内海笑了笑。接着我又聊起班上的女生说要去庆祝我脱单的事情,内海只是在一旁傻笑着听我吐槽。

“哦,说起来,朔他们也想帮我搞个脱单派对,可能也就是去KTV唱个歌,美月要不要一起来?”

“朔…さん”

“对啊…?哦,樱木朔(\japan{さくらき さく}),就是你认识的那个樱木学长。”

原来樱木学长的名字就是姓氏的前两个音,突然感觉这个名字有种既视感。

“那…不太好吧?都是学长的同班同学吧?”

“嗯,如果美月不想来就拒绝好了,就只是顺口一提。也不是什么大事。”

“学长想让美月去吗?”

内海喝了口鸡汤后说道:“我当然想咯!那几个人天天缠着我说要见你,再说就我们几个熟人去也没啥意思。而且朔也在,应该不至于难为美月。”

“唔…那就去吧?反正料理部一周也只用去一两次,美月比较闲。”

“嗯,时间可能会是文化祭后的第一个周末吧?因为文化祭前大部分人都有些事要忙。”

“好,知道啦。”

\newday{\cookie\milk\sunny}

闲聊着午餐时间转眼就结束了,所幸两个人终将整只鸡都清理得干净。虽说大部分都是学长吃的,我只能稍微帮他分担一些。清理好碗筷后,学长惯例在我家泡了杯咖啡,而我则把曲奇端到茶几,坐在沙发上边晃腿边品尝起来。我喝咖啡的话晚上会睡不着,所以学长只是递了杯热牛奶给我。

我:“学长真喜欢喝咖啡呢,每次都偷喝爸爸的咖啡!”

内海:“嘛,下次我会带同款咖啡过来的,让你爸发现就不好了…”

内海说着,就被我塞了块曲奇到嘴边。

我瞪大眼睛问道:“怎么样?!”

内海仔细地品尝后得出了结论:“好吃诶!这个形状也很正点,味道甜而不腻,这真的是美月第一次做吗?”

“呃…其实学长不用这样夸美月的…”

“啊?我说的是认真的呀?”

“好好好,就当学长是认真的吧!好吃就多吃点!”

内海一边喝着咖啡一边吃曲奇,看他享受的表情,看来第一次曲奇确实做的还不错?

我吃了一半的曲奇后,举着玻璃杯,脱掉拖鞋在沙发上屈起腿,侧靠在学长的肩上。

静谧的片刻后,内海突然问道:“所以美月的父母是做什么的?平时都不在家。”

我就这样靠着学长说道:“爸爸是专科医生,所以有时候急诊太多就会直接住在医院里。妈妈是护士,他俩工作时间基本上重合,爸爸睡在医院里的时候,妈妈担心他的身体,所以就想把精力放在他那边。他们年轻的时候就是在医院里认识的。”

正是因为父母都在医院工作,所以那天我才知道怎么处理跌打损伤,家里也总是备着各种医用物品。

内海:“可是这样谁来照顾美月呢?”

我叹了口气:“姐姐还在读高中的时候就是她一边学习一边照顾美月,等她去上大学了,我也读初三了,生活也能自理了。回想起来,很多菜也是从姐姐那学来的。”

内海:“诶?美月还有姐姐吗?”

“诶?难道以前没说过吗?嗯…不过美月和姐姐的关系也不是很好,爸爸对我们都很严,但是他对姐姐是当作男生的严厉,所有成绩都要追求最好,对美月就是当作女生的严厉,像是宵禁啊,不给买手机之类的啊。姐姐倒是刚上高中就给她买手机了!”我突然又想起什么,把脸侧向学长那边,“说起来,学长也没有手机吧?为什么呢?”

内海抿了口咖啡:“感觉没必要吧,反正也没啥非要手机才能联系的事情。即使在学生会,也都是能在办公室或者教室找到我的。”

我又将头靠在学长肩上:“唔…那幸好美月现在也没有手机,不然的话肯定也希望在手机上和学长聊天吧!”

内海:“所以美月和姐姐的关系为什么不好呢?不是她一直照顾着你吗?”

“就是因为教育方式不一样啦!虽然是一个家里的姐妹,但是性格却因此差得越来越多,尤其是感觉到姐姐的性格越来越像爸爸后,就不知不觉和她疏远了。其实也不能说不好吧,就是有点不知道怎么跟她相处。”

我喝了口牛奶,将目光眺望出窗外,“所以和爸爸的关系就那样僵着,和姐姐虽说也没那么僵,但是她已经搬到大学那边去住了,和关系最好的妈妈见面次数却不多。像今天这样有学长在身边就感觉稍微能够安下心来。学长…之后的每个周六也能这样过来陪美月吗?”

内海:“之前就答应过美月了吧,之后都会过来的。”

“那就好!”

我笑了笑,突然想到什么,将玻璃杯放到茶几上,转身环抱住学长。

“说起来,学长是怎么知道美月的生日的?!好像都一直没想起来问。”

内海憨笑着:“问了下佐藤さん,她跟我说过她查过你的资料。”

“啊?!那个坏女人!怎么还泄露个人隐私?!”我使劲摇着内海的肩膀,晃得他急忙先把手中的杯子放下。

内海:“哎呀,其实佐藤さん也没你想想的那么坏啦!”

“之前学长说要在文化祭帮佐藤さん的时候就觉得不对劲了!学长是不是还对她有留恋啊!”我气得更加卖力地摇晃着学长。

内海急得抓住我的手腕:“我和她一直都只是同事关系啦!那以后不提她了好吗?”

学长的力度并不大,感觉能轻易地挣开,但我就这样任由他抓着。

“哼…那把学长的生日也告诉美月,这事就算过去了。”

内海总感觉自己吃了个哑巴亏,不过也不是什么需要特别隐瞒的事情:“圣诞节,十二月二十五日。”

我:“这么冷的时候出生?难怪不怕冷诶!”

内海无奈地笑笑:“我倒是觉得和生日没太大的关系。”

冬天的生日,心中突然有了一个想法:“那学长就好好期待一下生日礼物吧!这次绝对不会让学长提前知道的!”



这本是一个恬静的午后,正热恋着的情侣随意打闹的秋旭时光,却因突如其来的开锁声,变得心惊胆战起来。


“我回来啦!…诶?家里有客人吗?…美月!在家吗?!”


进门的是一位穿着浅色牛仔裤,外面批着棕色大衣的年轻女性。

女性:“美月你在家就打声招呼啊!…呃,你是谁?”

因为事情发生的太过突然,我反而是躲在了学长的身后,完全不敢与姐姐对视。

内海像是被枪指着般举起双手:“我是…美(\japan{み})…雨宫さん的朋友!”

“美…?”雨宫姐姐仅从一个音便察觉到了一丝异样,“先不管你,美月,你能先出来解释一下情况吗?”

我急速揣摩着怎么才能不让姐姐起疑心,悄悄地从学长背后探出来,用着虚伪的笑容问道:“那个…姐姐大人,你怎么回来了?”

姐姐:“什么姐姐大人,你吃错药了吗?妈妈说我今年都没怎么回过家,所以今晚一起回家聚一聚,他们没跟你说吗?”

我委屈道:“没说啊!”

姐姐:“哦…虽然跟不跟你说都没啥关系,毕竟你一直在家来着。”

我:“那也要说一声啊!妹妹的人权呢!”

“妹妹的人权不就是趁家人都不在偷偷带男朋友回家卿卿我我吗?”姐姐随后用犀利的眼神看向内海,“我有说错吗?男友くん(\japan{彼氏くん})?”

内海:“那个…我是不是不该出现在这里?”

我一把环住学长的脖颈:“男友咋了?我就交男友了,比你这个单身姐姐有女人味多了!”

姐姐用手指绕过内海指着我,愤怒道:“不是,我单身?我单身是因为我想单着,想追我的男的都能从大学排到家门口了!你现在翅膀硬了是吧,那要不要我把这事告诉爸爸啊?”

一听到姐姐要告状,我慌张地双手合十哀求道:“姐姐大人息怒,妹妹知错了,只求您不要把这件事告诉爸爸。”

内海一直夹在姐妹中间不知所措。

姐姐叹了口气:“好了,不跟你闹了。等下你出去买点晚餐要用的食材,我就帮你保密。”  

“唔,好…”

随后她便转身找纸笔写购物清单。我从沙发上下来,拉着学长的袖子准备把他送走。

内海悄悄凑到我的耳边说道:“好尴尬…”

我:“别慌…一切都在美月的掌控之中。”

不一会,我们已经在玄关处穿好了鞋,姐姐写好清单后就走到玄关递到我的手上:“哦对了,购物美月你一个人去就行了,男友くん你给我留下。”

我在原地愣了一会,确认了自己没有听错后惊呼道:“哈啊?”

姐姐:“哈啊什么,我只说了让你一个人去啊?”

内海:“可是我留下不太好吧?”

姐姐转向内海问道:“男友くん想让美月被她爸爸关禁闭吗?不想的话现在你俩就得听我的,明白吗?”

内海看了看我,不知道该如何是好,我也没办法, 反正只要爸爸不知情就能从长计议。

我悄悄在内海的耳边轻语:“先帮忙稳住姐姐吧…等美月回来…”

内海听罢点了点头。

“不准对学长出手。”我最后像姐姐叮嘱了一句后,带着钥匙和购物袋出门了。

\cutlinef{走前学长把夹克披在我身上}

在一阵关门声后,雨宫姐姐捂着额头终于忍不住问了我一句:“男友くん,你不觉得美月有点呆吗?你是不是什么地方被她骗了?”

我一脸疑惑:“诶?这是雨宫さん可爱的地方吧?”

“我看你也有点问题。”雨宫姐姐转身走进屋,“你也进来吧,我有点事要问你。”

我只好再次脱掉鞋子进屋。

“看来你应该不用泡茶了吧?我看你已经有一杯咖啡了。你就像刚才那样坐沙发那吧。”说着,雨宫姐姐到冰箱里拿了些牛奶倒进杯中喝了起来。

我走到沙发边跪坐着:“那个…雨宫姐姐喝冰牛奶不会肚子疼吗?”

雨宫姐姐呆呆地看了看手中的牛奶,又看了看我:“哦…美月好像是会肚子疼来着。不过确实最好先热一下…原来你连美月的这些都知道了吗?”雨宫姐姐转身将玻璃杯放入锅中水浴加热。

我:“嗯…就是感觉雨宫姐姐和雨宫さん有很多不一样的地方。”

雨宫姐姐:“嘿诶?不一样的地方?那先说说有没有一样的地方吧,我对这个更感兴趣。”

我:“比如…都很强势?”

雨宫姐姐:“等等…强势?你确定你在说相同点吗?美月很强势?”

我:“嗯…她总是会在一些意想不到的地方特别主动。”

“男友くん,你绝对是被她骗了。美月绝对不是这样的人。”雨宫姐姐热好牛奶后也来到茶几边跪坐下,“哦还有,忘记自我介绍了,我叫雨宫天月(\japan{あめみや あづき}),是美月的姐姐。”

“内海侑(\japan{うつみ ゆう}),雨宫さん的学长。”

“只是学长吗?”天月露出期待的表情喝了口牛奶。

“现在…姑且也是她的男友。”

天月咯咯地笑着:“在我面前不用拘束,你想叫她美月就叫吧,我毕竟是她的姐姐,不是她的爸爸,我又不是什么坏人。”天月顺手从茶几上的盘子里取出一片曲奇吃了起来,“哇,这曲奇真好吃,是你给美月做的吗?”

我摇摇头:“这是美月做的。”

天月难以置信地看着眼前的曲奇:“诶?她现在还会做曲奇了吗?天哪,明明我教她做菜的时候她都懒得学,这还是我认识的美月吗?…诶不对,这是她给你做的吧?天哪…看来你真是她男友啊?”

我不知道是不是要告诉她,呆或许也是她们的相同点。

天月:“哎呀,我开玩笑的啦!其实我就是有些震惊,她居然会因为谈恋爱而改变这么多。”

我腼腆地笑笑:“她是个很出色的女生,即使没有喜欢上我,她也会有主动去改变的一天的。”

“你也太迁就美月了,这对她可不好。”天月看着我,露出一副欣慰的笑容,“另外以后你就跟美月一样叫我姐姐(\japan{お姉さん})吧。”

我一脸疑惑地看着天月。

天月:“难道不愿意吗?这可是认可你这个男友的意思哦!”

我:“会不会…不太好?”

“不叫的话我就把你的事告诉她爸爸了哦,你可能会被他打进医院哦?你要知道你的腿被打断了,他也能帮你治好,然后再打断一次…”天月的手上模拟着打断腿的动作。

“…她爸是骨科的吗?”

“那倒不是,所以才更恐怖啊!以后你的腿可能就不能走路了!”

“知…知道了…”我倒不是怕被打断腿,只是美月叮嘱了要我稳住她,“姐…姐姐?”

天月:“嗯,侑くん!…我可以这么叫你吗?”

我:“啊…一般我朋友都是叫我内海的,侑只有一个音稍微有点太亲昵了。”

天月不以为意:“哦,那就是我可以叫你侑くん吧!”

“姐姐的话…倒也无所谓。”

天月又喝了口牛奶,拿起最后一块曲奇,用着审视的语气问道:“侑くん,你喜欢美月哪点?”

“温柔的点,善解人意的点,主动的点,黏人的点。”说着,脑海里与美月的回忆就慢慢浮现在眼前,“哦,还有爱哭的点,吃醋的点,生气的点,霸道的点…”

天月看着少年时而会心一笑,时而宠溺地说着妹妹的坏毛病,无论好的点还是坏的点,他都一五一十地说了出来。

“好了,够了…太长了懒得听。”天月像是在吃狗粮般赶紧打断我,“但是你说她爱哭是吧,难道你经常把她弄哭吗?”

我:“唔…她有哭过几次…虽然也不能说跟我没有关系…”

“哦…那你听好了,要是以后再被我知道你弄哭她,你就别想再见到她了,明白吗?”天月的这句话冰冷地感觉不到一丝温度,用机械般的口吻划清了她对我的最终要求。说罢,她的眼神又变得如开始般友善起来:“嘛,毕竟她是我的妹妹,她的男朋友我可是要求很高的!”

“我…会努力的。”

天月:“好啦,时间也差不多了,那侑くん把联系方式给我吧,以后我会定期来找你问美月的近况的。”

我有些尴尬,因为:“我没有手机。”

天月一脸震惊:“啊?我还以为现在的高中生只有美月没有手机了…”

我:“那个…就是稍微有些情况吧,而且手机对我也确实没什么用。”

天月像是突然察觉到什么问道:“说起来,你们的学校周六不用上课呢。明明是以升学为主的高中?”

我:“嗯,因为前几年有有家长闹说周六的课上了也是浪费时间,不如空出来去上补习班。”

“那倒确实,课上教的那些补习班都会讲。”天月点点头,“所以说,你是美月的学长,那至少高二了吧?你不用去上补习班吗?”

“算是有在上吧?我有个朋友会把他补习班上每周的资料借给我,我抽些时间自己准备就好了。”

“诶?居然是自学吗?难不成侑くん成绩很好?”天月有些吃惊。

“没有,只是这种方式比较适合我。”我有些不知如何开口。说起来,这些事我也从来没和美月说过,也幸亏她从来没问起过。

天月看出我有些为难,也不再过问了:“好吧…每个人都有难处吧。说回来,你和美月都没有手机到底是怎么谈到一起的?你们活在上个世纪?”

“总归学校里还是能见到的嘛。”我喝了口咖啡,紧跟着毕恭毕敬地问道,“但是姐姐要问美月的近况…为什么要找我呢?”

天月叹了口气:“因为我跟那孩子关系有些微妙。感觉侑くん现在应该比我更了解美月了吧。”天月看了看手机:“唔,美月那边应该也快要结账了,侑くん,你去帮美月拎一下东西吧,她一个人估计够呛。其他的我们下次见面再聊。”

我应了一声,起身走到玄关,天月跟在后面也来到了玄关:“你只要把她送回来就行了,今天就别再进我们家,直接回家去吧,不然等会再被她爸撞见就麻烦了。”

我点点头,正转身握着门把手,想了想有些话还是应该告诉姐姐,于是又转回来看向姐姐:“姐姐,我觉得你也稍微体谅一下美月会比较好,我想美月她还是打心底里喜欢你的。”

天月被突如其来的话语震撼到:“诶?难道她还跟你说了她喜欢我吗?”

我摇摇头:“因为她每次提到和姐姐你关系不好的时候,都是唉声叹气的。所以我想,她还是想和姐姐变得要好的…就只是这样的猜测而已。”

天月的嘴角悠悠上扬:“这样啊?没想到有一天我作为姐姐还要被一个外人来教育哇?”

“没,只是一点个人的想法,可能有些失礼了…那我就告辞了,再见姐姐。”我最后向姐姐鞠了一躬,走出了雨宫的家。

“好,路上小心,男友くん。”

\cutlinef\turned

“诶!学长怎么跑出来了?姐姐放过学长了嘛?”

我从雨宫的家里走出来后,一直来到旁边的小商场,果然在这里碰到了结完账的美月。

美月:“居然有这么多东西,学长帮忙拿一下吧!”

我顺手接过美月手上的一大袋购物袋,确实有些沉,美月一个人拎肯定会累得半死。因为我只穿着一件纯棉的长袖T恤,美月便想把夹克还给我,但是和我牵手比对了下温度后,她也明白或许她才是更需要这件夹克的人。

美月拉着我往回走,我趁机偷偷瞄了眼手中的袋子,里面除了食物还有几包卫生巾。这么说来,这三周的相处下来,她除了喝牛奶着凉拉肚子外,确实都没有出现过什么异样。

回家的路上美月便开始了抱怨:“怎么要买这么多东西,一晚上肯定吃不完!姐姐还要美月一个人来买,太坏了!所以她跟学长说了什么没?!”

我一只手拎着购物袋,另一只手被水母缠着:“其实也没什么,随便聊了聊,像是美月做的曲奇之类的。”

美月一副期待的样子:“诶?姐姐吃了曲奇吗?她怎么说?”

“她夸奖美月做的好吃呢!”

“嘿诶?一点也不高兴呢,她估计是当成是学长做的了吧,想要讨好一下学长。”虽然美月嘴上说着嫌弃,脚步却轻快了起来。

“哪有,我跟她说了是美月做的曲奇,她更加惊喜了。”

“哼哼~这样啊,那毕竟是美月做的曲奇!”

我也喜欢美月现在这样天真的样子。

“美月的姐姐看起来很喜欢美月呢。”我虽然也想抱抱美月,但是两只手实在腾不出空间。

“学长在说谁?是在说雨宫天月吗?这是怎么组成一句话的?”

“嗯,就是美月的姐姐。怎么说呢?跟美月的姐姐聊了一会后发现,你们姐妹都是傲娇呢。明明都关心对方,但都不想第一个向对方示弱。所以就只能通过外人来了解彼此。美月的姐姐对我说要我照顾好你,还威胁我不准欺负你,又一边关心你最近的变化。”

“姐姐…她真的跟学长说这些了吗?”美月有些不知所措,像是第一次了解到姐姐的另一面。

“嗯,这种感觉我也有体会。越是亲近的人,就越是会变得不坦诚,变得固执己见,渐渐地失去认真交流的耐心,最后或许会变得恶语相向。”

“那不是很糟糕吗…?”美月低头思考着什么。

“但是我觉得这就是家人吧,正是因为你们是姐妹,所以才会有恃无恐,即使真的伤害过彼此,最后也会变成过往云烟一笑而过。因为美月一直都喜欢着姐姐不是吗?”

“美月…才不喜欢姐姐,姐姐最差劲了。”美月狡辩着。

“嗯,美月是那种「正是因为很喜欢才会开始讨厌」的类型对吧?就跟交往那天美月跟我说的话一样。”

“干嘛!学长是学长,姐姐是姐姐!…”

美月话音刚落,我便说出了我的结论:“但是心情是一样的对吧?”

秋日缓缓下落,两人的影子被越拉越长,将突然打断的惬意午后稍稍延续。或许也因此,女孩稍微变得坦诚了一些。

“可…可能吧…?”美月嘟起小嘴,“跟学长说话总是被带着跑。讨厌!”

我把美月送到家门口:“那我也该回去了,周一再见?”

美月转身抱着我,不舍得我的体温:“学长…美月有点怕见姐姐。”

“因为意识到自己喜欢姐姐吗?”

美月在我怀里点点头:“嗯,而且她要是也关心美月的话…感觉有点对不起姐姐…”

我用手摸着美月的后脑勺:“这份心情,美月的姐姐也是一样的吧?毕竟你们是姐妹嘛!”

美月从我的怀里抬起头来:“真的吗?”

“相信我,试着和姐姐和好吧?”

“那…给美月一点勇气吧…?”

我明白美月眼神的含义,稍微俯下身子,美月也情不自禁地轻踮脚尖。

我离开她的唇,浅笑着问道:“勇气传达到了吗?”

“嗯,快满出来了!(\japan{いっぱい!})”美月红着脸莞尔道。

美月将夹克还给我,我最后捏了捏她的脸转身离开。

如果她们能和好就好了。

\cutlinesm

\newday{\sunny}

从美月读幼儿园时起,我就开始每天小学放学后来接美月回家。偶尔她想吃棒棒糖,我就带她去路上的超市。她想和其他同龄人玩,我就带她去附近的公园。即使她只想和我玩,我也会陪她一起玩。她发烧我请假在家照顾她。有时爸妈不回家吃饭,我得为她做饭。我做的饭我自己都觉得难吃,她却总是夸我做的饭香。

“姐(\japan{ねえねえ})!看!城堡!”

某天她正在沙堆里堆着完全看不出是城堡模样的玩意朝我。

“美月,你觉得那是城堡吗?”

“是啊!不然还能是什么?”

“哎…”我叹着气,美月比我小五岁,所以有时候我会觉得她太过幼稚,“城堡要这样,再这样…你看这样是不是更像城堡了?”

美月的双眼发着光看着我:“哇!姐是天才吧!姐什么都会!”

我无奈地向她露出微笑,看着妹妹天真无邪的样子,自己也会跟着幸福起来。

等她玩腻,我帮她洗手洗膝盖,帮她擦鼻子,然后领她回家。经常也会有其他家长说我比起她的姐姐,更像她的妈妈。

“美月晚上想吃什么?妈妈说我们可以买点自己想吃的回去她来做。”

“诶!胡萝卜!”

“然后呢?”

“唔!…叫不出来。”

“呃…那一起去超市看看吧。”

我牵着她的手,她总是一蹦一跳的。

走到半路上,她突然停下脚步,躲到我的身后:“姐…那里有只狗…”

她因为前段时间被狗追过,所以现在怕狗怕得要命。

“姐在,不要怕。”

它看到美月一副做贼心虚的模样,便吼了起来。

“呜呜…姐!它在叫!”

我也凶狠地盯着那条狗:“哪来的野狗,狗叫什么!”

尽管那狗听不懂,但它还是被吼得安静了下来。

等我们经过它,它又开始狗叫起来。

美月害怕地抓紧我的手,我忍无可忍,捡起路边的树枝朝它追去:“本天月的妹妹你都敢吼,真当自己是个畜生了!”

我追着直到狗跑出十几米远,身后某个小生物才追上来抱住我。

“姐…不要丢下我一个人…”

我扔掉树枝拍拍美月的头:“姐不就在这吗?狗已经赶跑了,我们回家吧。”

我再次牵起美月的手。她紧紧握着我的手,轻声问道:“姐以后能一直陪着美月吗?”

“当然,毕竟我们是姐妹啦。”

她露出幸福的笑容:“太好啦!最喜欢姐啦!”

\cutlinesm

\newday{\cmoon}

美月:“姐!姐!能借你的SWITCH玩吗?!”

那是美月还在读小学二年级,我刚升上初一的时候,我如往常一样回到家,她如往常一样跑到玄关来扑到我身上,像一只惹人怜爱的小兔子。

我将美月抱下来:“直接送你了吧,我以后应该没什么时间玩了,等下还得去补习班。”

美月的手抓着我的袖口:“姐!那周末能一起玩吗?”

我:“周末要去练吉他。抱歉了美月,爸爸给我安排了很多课,爸妈不是说要带你去水族馆吗?有机会再陪你吧?”

“周六他们也只是把我丢在水族馆,要回家了才来找我…还有还有!”美月想起什么似的在我身上一蹦一跳的,“上周我遇到一个男孩!他好像也是孤零零的!而且他叫我姐姐!我现在也和姐一样是一个姐姐了哦!(\japan{あたしは姉姉みたいなお姉さんになるよ!})”

美月露出一个甜蜜的微笑,不久前掉的门牙槽里已经能看见新牙。

“嗯嗯,那不是挺好的吗?美月多找同龄人玩会更有共同话题吧?”我将美月推开,往房间走去。

美月不依不挠,跟在我的身后一直进到我的房间:“姐…能不能抽空陪陪我,爸爸妈妈今天也都不在家。晚上能和姐一起睡嘛!?”

“算了吧,我回来的会有点晚,美月一个人早点睡吧。然后把SWITCH拿走吧,一个人的时候就打打游戏打发时间吧。都一样的。”我指了指桌上的游戏机,然后开始准备补习班需要的课本,“好了,别待在我房间了,我等下就要出门了。”

“好…”美月拿起桌上的游戏机走出了我的房间。

我收拾完东西走出房间,美月已经赖在沙发上玩起了游戏。听声音,她应该在玩最近沉迷的海盗冒险游戏吧。

美月转头看过来大喊道:“姐!路上小心!”

我回头看着无忧无虑的她,却只能感觉到手中书包的重量。

一切或许就是从那天开始的吧?从那天起,我开始有些讨厌妹妹。

\cutlinesm

一直到她升入初一,那时我读高三。妹妹因为戴上牙套,在学校被同班女生霸凌。本来计划等我毕业搬新家的爸爸,一怒之下选择让妹妹转校。在无视我的感受后,我们一家搬到了现在这个一户建。

那天,我如往常一样正在回家的路上,电车驶向的却不是往常的方向。

站台上,我百无聊赖地刷着手机,旁边的同学熙熙攘攘地谈论着等下去哪个咖啡厅,那对穿着同一套校服的兄妹,哥哥正在把撒娇的妹妹推得远远的。我已经不在意这些是否曾经出现在我的生活之中,唯一可以确定的一件事是,这些都与现在的我无关。因为我的课余生活,随着升上高三变得只剩下了补习班。我戴上耳机,将聒噪隔离。

电车停了一站又一战,那些与我同一校服的人都渐渐不见了踪影。我看着车门窗上疲倦的自己,眼神中只剩下空洞,而空洞之下,是无处释放的压抑。

我回到家时,一听到我的声音,美月也如往常一样来到玄关接我,但是能看出她的神情与动作都因霸凌的事情,开始变得心惊胆战。

美月:“姐姐,欢迎回来。(\japan{お姉さん,お帰り。})”

我无视了她,径直往房间里走去。

美月则是跟在我的后面:“姐姐…晚饭做好了,等下一起吃吧?”

我摇摇头:“不了吧,我等下就要去补习班了,时间有点紧,路上买个三明治吃就行了。”

\newday{\hate\cmoon}

美月:“哦对,姐姐最近回来都变晚了呢…那就没办法了…”

我终于有些忍不住,在我的房间门口前停下:“哦…?哦什么哦?”

美月由于急停差点撞到我身上,赶紧在我的身后停下:“诶?什么?”

我突然觉得莫名地可笑:“你知道我为什么会回来晚吗?你知道这个新家离我的学校有多远吗?我每天都要多花一个小时在通勤的路上,而这都是因为那个混蛋老爹为了你搬到这个鬼地方来。我跟他说这会对我的学业有影响,你知道他怎么跟我说吗?他说:「你是姐姐要让着妹妹」。然后你现在就只是说一句「哦,那就没办法了呢。」你不觉得你很幼稚吗?”

美月连忙解释:“我…我没有那个意思…姐姐…”

我转过头来,恶狠狠地盯着自己曾经宠爱的妹妹:“而你就只需要每天这样待着家里,过着无忧无虑的日子,不是吗?你知道的吧,我从小学开始就上着各种各样的班,从来不能有怨言,只能被爸爸一直逼迫着去学那些不知道有什么用的乱七八糟的东西。”妹妹因为突然的嘶吼开始啜泣,而我也因此变得越来越大声,“我初一也被别人霸凌,就因为我在课间看言情小说这种无聊的小事,而我回到家后,爸爸只是一个劲地骂我「没用,为什么不霸凌回去,就是因为我软弱,才会被别人欺负」!所以我第二天就朝那个人的鼻子揍了一拳,把她的鼻梁打断了,血不断地从她的鼻子里流出了,我至今都能记得那个场景。只有那个场景才能让我清楚地意识到没有人会帮我,即便是自己的父亲!从那之后再也没有人敢欺负我。”

“但是你呢?你小学去上钢琴课,上了一个月说没意思爸爸就帮你退了课。你说没有想学的东西,爸爸就开始让着你,就连他抽烟都是因为你讨厌才戒的,明明我也很不喜欢他抽烟啊!现在你被别人欺负,爸爸就会来替你出头,爸爸可真是偏心啊!我没有自己的时间,没有自己的生活,不能有自己的喜好,这些东西全部都被爸爸给夺走了,这些东西爸爸全部都让给了你!你明白吗?爸爸的偏心,我的爱好,我的自由,我的一切都已经被你夺走,就只是因为\highlight{你是妹妹,我是姐姐!}而你要做的,就只是在这哭哭啼啼的吗?哭到底能解决什么问题啊?!”

她的眼泪如珍珠般滚落,半天说不出一句话。

我失望地看着她,转身打算进入房间,却被她拉住手:“我…我只是想要姐姐陪…想要爸爸妈妈在家里…想和朋友一起玩…我做错什么了?我到底做错什么了啊!…我也想让姐姐开心!但是每天姐姐在家的时间就那么一点啊!…为什么大家都不关心我的感受!爸爸那种自以为是的偏爱,我根本不想要啊!我想要你们陪在我身边啊!”

我明明也想陪在她身边,可我又能做什么呢?果然有其父必有其女吧,因为率先浮现在我脑海的画面,就是父亲对着我恶语相向的画面:“\highlight{根本没人关心你,美月!}快点长大吧!”

我为自己感到可怜,我为自己说过的话感到恶心,它让我伤害了我最宠爱的人,它让我变成了我最讨厌的人。

半夜从补习班回到家,我敲响美月房间的门。她出来后,我为傍晚的冲动向她道歉。她微笑着原谅了我,她说理解我,说是她太幼稚了,她的眼神里充满着寂寞与无助。从那天起,美月封锁了自己的内心,她再也没有对我笑过。

\cutlinesma

\newday{\hate\cloudy}

某个春日的清晨,父亲把车后盖合上,朝着我们吆喝:“好了!都收拾完了,那天月该上车了,准备去东京大学了!”他看了看手表喃喃道,“要开蛮久的,希望能在半夜前回来。”

母亲:“我们一家都没去过东京大学呢!今天能蹭蹭天儿(\japan{あっちゃん})的面子呢!月儿(\japan{みっちゃん})也一起去吧?”

美月站在家门口,一只手抓着另一边的臂膀,有些不知所措:“不了吧,妈妈,我对大学也不感兴趣。”

“嗨呀,月儿不想去就算了吧,天月、孩子妈,上车!”父亲吆喝着坐进了主驾驶。

母亲看着美月,又看看我,叹了口气:“哎,那月儿,一个人在家乖点哦~”

母亲走到倚在副驾驶门前的我面前,轻声说道:“做姐姐的跟妹妹好好道别,别板着个脸,看着我就难受。”

母亲拍拍我的后背,把我推到美月的面前,美月却低下头去,不与我对视。

“那个…美月…”我挠挠后脖颈处的头发,不知所措的不止美月一个人,“那个…姐姐要去东京住了,以后家里就你一个人了…好好照顾自己,好吗?”

美月耻笑了一声,随后又恢复了无表情的状态:“姐姐才是,晚上不要再吃三明治了。”

“嗯…我会记住的。”

话音刚落,空气便陷入了沉默。或许这样的对话就算是告别了吧,或许这样就够了吧?

我转过身,再一次从美月的眼神中逃开。我打开副驾驶的门,坐了进去。

母亲坐在我的身后一副头疼的模样:“真没个姐姐样。”

我:“我…只是不习惯煽情而已!”

父亲:“哎呀,天月考上东大就已经是「姐姐样」了!孩子妈在担心什么啊?那我们出发了,孩子妈,跟月儿说一声。”

汽车的引擎声响起,母亲伸出车窗外,向美月挥挥手,美月将手微微举起,有些不甘愿地与我们告别。

我没有挥手,因为我没有资格。

汽车渐行渐远,慢慢将破碎的过往一点点剥离出自己的身体。

突然,后视镜里的那个身影跑到了街道上,用着夸张的幅度挥着手。她没有说一句话,像是无声的告别。

我仿佛突然间意识到什么,她天真幼稚并不是向来如此,她天真幼稚只因为她是我的妹妹。所以即使被我狠狠伤害过,她仍然不舍得我离开。

直到我注意到自己流下了眼泪,我才明白:我并不讨厌她,我讨厌的是她眼中的自己。

\cutlinesmb

\newday{\love\sunny\sunset}

美月铆足了劲把购物袋挪到玄关,我走到玄关处:“欢迎回来,美月。把袋子先放玄关吧,先拿点东西出来。”

美月点点头:“晚餐要做什么?”

我:“咖喱,打算炖的久一点,美月来帮忙吧?”

美月跟着我来到厨房处理食材。

我想起刚才在窗口偷窥到的场景:“啊啊~真是便宜那小子了啊,随随便便就能亲我的妹妹!”

美月呛了一口水诧异地看向我:“噗,什…什么?姐姐在说什么啊!”

我朝着美月笑了笑:“侑くん,是个好孩子呢。”

美月害羞地低下头去:“其实最开始是我先亲他的…”

“哦!难道这就是侑くん说的美月霸道的地方?”

“啊?!什么霸道啦!一点都不淑女!学长还跟姐姐说什么了?!”

“夸了美月很多哦,我都没想到美月有这么多优点。”

“他说什么了!他说什么了!”美月冲到我面前急不可耐地想知道答案。

我轻轻挡着她以免她又粘人:“自己去问他啦!这种事情还要姐姐来转述吗?!”

“怎么好意思问啦!真是的,姐姐老是这样!还有,不要叫他侑くん,我不喜欢姐姐那样叫他。”

我看着美月着急的模样,忍不住调戏她:“那要不我也像美月一样叫他学长?”

美月:“不行!学长只有美月才能叫的!姐姐都比学长大三岁,叫他学长是想显得自己年轻吗?”

我:“呃…我觉得有必要把你现在的样子拍下来,下次给你的男朋友看。”

我假装着拿出手机拍下美月现在的表情。

美月:“什么样子?说真话的样子?”

我按下拍摄键,不怀好意地笑着:“美月吃醋的样子~”

美月瞬间红通了脸,蹦着想要抢过我的手机:“什么啊!姐姐要干嘛啊!不要给他看!”

我举起双手表示投降:“好好好,不给他看。让姐姐拍几张妹妹的照片留着自己看可以吗?”

美月这才停下来:“怎么突然要拍我?我有什么好拍的?”

“因为以前都没注意到自己妹妹变得这么可爱了~”

“呃…姐姐倒是一如既往的烦人。”美月露出一副嫌弃的模样,说完便侧过身继续洗着菜。

我自嘲般笑笑,继续切着胡萝卜:“但我还挺喜欢的,美月现在的样子。”

美月低着头洗着菜,看不清她的表情。不一会,她将洗好并削皮的土豆递给我:“干嘛,突然这么肉麻。”

我接过土豆,不知道为什么,心里痒痒的。听了侑くん的话后,我突然在意起美月的真实想法。

“因为美月那时候说了吧,我从没了解过美月的感受。”

“所以呢?”

“因为以前做了一些与姐姐不相称的事情…所以…所以说啊…美月现在是怎么看待我的呢…?”

“嘿诶~姐姐原来在意这个吗?”水槽那边传来撩拨的声音,“但是姐姐真的想听吗?”

“想听…就算是骂我也行。只要是美月真实的想法,请告诉姐姐吧。”

“嗯…姐姐刚才说了喜欢美月吧。那我的感受…”她的声音变得轻松愉悦。

我嗯了一声,朝她看去,她也满心欢喜地望向我,如同当初跑到玄关接我时的小孩模样。

\cutlinef{“当然是最喜欢姐姐啦!”}

\newpage


\newday{\sunny}


\character{雨宫 天月(あめみや あずき)}{
	\高光{出生日期}&  &\highlight{2003年7月23日}\\
	\高光{身高}&&\highlight{166 cm}\\
	\高光{体重}&&\highlight{53 kg}\\\hline\hline
	\高光{喜欢的颜色}&&\colorfy{purple}{\color{white}{紫色}}\\
	\高光{喜欢的食物}&&布丁\\
	\高光{喜欢的运动}&&网球\\
	\高光{喜欢的事物}&&妹妹\\
	\高光{讨厌的事物}&&让妹妹不开心的存在\\\hline\hline
	\高光{日常装饰}&\高光{发型}&过肩发\\
	&\高光{装饰}&熠熠生辉的太阳耳钉\\
}
{
	喜欢和讨厌的事都和我有关?
	
	那当然啦!毕竟是我写的嘛!
}


\newpage



